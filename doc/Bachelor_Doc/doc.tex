\documentclass[czech, kiv, ba, he, iso690alph, pdf]{fasthesis}

\usepackage{url}
\usepackage{enumitem}

\title{Implementace modulu pro import údajů RÚIAN}
\author{Martin}{Schön}{}{}
\supervisor{Martin Bíkl, Ing. Petr Přibyl, Ing. Martin Zíma Ph.D.}
\stagworkid{100718}
\assignment{figures/zadani.pdf}
\signdate{31}{12}{2024}{V Plzni}
\addbibresource{bib.bib}
\abstract{Text abstraktu v~jazyce práce, tj. zde česky.}
{The abstract text in a secondary language, here in English.}
\keywords{
    RÚIAN -- Registr územní identifikace adres a nemovitostí\\
    VFR  -- Výměnný formát RÚIAN
}
\acknowledgement{Text poděkování.}

\begin{document}
\frontpages[tm]
\tableofcontents
%---------------------------------------------------------------------------
\chapter{Úvod}
%TODO: Úvod

\chapter{RÚIAN}
\section{Co je to RÚIAN}
Jedná se o~státní informační systém v~České republice, který obsahuje 
informace o~adresách, budovách, parcelách a~dalších objektech. Systém 
je spravován Českým úřadem zeměměřickým a~katastrálním (ČÚZK). 
Data jsou využívána v~mnoha oblastech, například v~urbanistickém plánování, 
geodézii, nebo při správě nemovitostí. Jednotlivé prvky jsou zobrazovány na~mapách 
státního mapového díla a~digitální mapě veřejné správy.

Data z~RÚIAN jsou veřejně dostupná a~lze je získat z~webové služby na~adrese 
\url{https://vdp.cuzk.cz/vdp/ruian}. V~této aplikaci je možné vyhledávat konkrétní 
prvky nebo ověřit jejich existenci. Data jsou poskytována ve formátu XML a~lze je 
automaticky stahovat pomocí API. Tyto informace jsou pravidelně aktualizovány, 
což zajišťuje jejich aktuálnost a~přesnost.

Cílem této práce je výměnný formát z~této služby odkud je třeba stáhnout, 
zpracovat a uložit do databáze.

\section{VFR}
Výměnný formát RÚIAN je služba poskytující data.
Je možné stahovat data dle zadaných formátů: Standardní, Historický a~Speciální.
Každý formát obsahuje dodatečně parametry, které je možné nastavit.
Data z~VFR jsou ve~formátu XML.
Každý element obsahuje atributy, které obsahují informace o~dané entitě (Tabulce).
\pagebreak
\begin {itemize}
    \item \textbf{Standardní} -- obsahuje úplná nebo přírůstková data.
    \begin {itemize}
        \item Časový rozsah: Přírůstky od~data / Úplná kopie
        \item Územní prvky: Stát až~ZJS / Obec a~podřadné
        \item Datová sada: Základní / Kompletní
        \item Výběr z údajů: Základní údaje / Gen. hranice, Originální hranice, Vlajky a~znaky
        \item Územní omezení: ČR / Kraj (VÚSC) / ORP / Obec
    \end {itemize}
    \item \textbf{Historický} -- obsahuje historická data.
    \begin {itemize}
        \item Časový rozsah: Přírůstky od~data / Úplná kopie
        \item Územní prvky: Stát až~ZJS / Obec a~podřadné
        \item Územní omezení: ČR / Kraj (VÚSC) / ORP / Obec
    \end {itemize}
    \item \textbf{Speciální} -- obsahuje speciální data.
    \begin {itemize}
        \item Časový rozsah: Přírůstky od~data / Úplná kopie
        \item Výběr z~údajů: Číselníky / Vazby / Vazby a číselníky
        \item Kategorie: Všechny / Geodetické body / Nerostné bohatství
    \end {itemize}
\end {itemize}

\section{Tabulky}
Data z~RÚIAN jsou rozdělena do několika tabulek.
Jak je vidět na~obrázku~\ref{fig:ruian_tables}, každá tabulka obsahuje jiné informace.
Některé tabulky obsažené v~RÚIAN jsou nepotřebné. 
Příkladem může být tabulka \textit{Stát}, která obsahuje informace o~státu Česká republika.
Ovšem RÚIAN obsahuje pouze informace o~České republice, tudíž tato tabulka je nadbytečná.
Definice, které tabulky budou zpracovány a~které budou ignorovány, záleží na~specifikaci v~XSD (XML Schema Definition) souborech v~dokumentaci VFR. 
\pagebreak

\begin{figure}[ht]
    \centering
    \includegraphics[width=\textwidth]{figures/ruian_tables.png}
    \caption{Tabulky RÚIAN}
    \label{fig:ruian_tables}
\end{figure}

\section{Uložení dat}
Vzhledem k~formátu dat z~VFR, je~potřeba vybrat vhodný způsob uložení dat.
Nejlepší možností se~tedy nabízí Databáze.
Databáze umožňuje ukládání, organizaci a~vyhledávání dat.
Data mají pevnou strukturu a~jsou vzájemně propojena klíči.

\chapter{Databáze}
\section{Co je to databáze}
Databáze je software, který umožňuje ukládání, organizaci a~vyhledávání dat.
Existují dva hladiny databází: relační a~NoSQL.

\textbf{Relační databáze}
\begin{itemize}[itemsep=-2pt]
    \item Data jsou uložena v~tabulkách
    \item Každý řádek tabulky obsahuje jeden záznam
    \item Každý sloupec tabulky obsahuje jeden atribut
    \item Vztahy mezi tabulkami jsou definovány klíči
    \item Využití SQL jazyka
\end{itemize}
Relační databáze jsou vhodné pro strukturovaná data, která mají pevnou strukturu.
Jedná se o~nejčastěji používaný typ databáze.


\textbf{Objektové databáze}
\begin{itemize}[itemsep=-2pt]
    \item Data jsou uložena jako objekty
    \item Každý objekt obsahuje atributy a~metody
    \item Vztahy mezi objekty jsou definovány referencemi
    \item Využití objektově orientovaného jazyka
\end{itemize}
Objektové databáze jsou vhodné pro nestrukturovaná data, která mají složitou strukturu.
Jedná se o~novější typ databáze, který je vhodný pro moderní aplikace.

Jaká je tedy vhodná databáze pro uložení dat z~RÚIAN VFR?
Vzhledem k~tomu, že data z~VFR mají pevnou strukturu a~jsou vzájemně propojena klíči, 
je nejlepší volbou relační databáze. Pro zpracování dat z~RÚIAN VFR je tedy vhodná relační databáze.
Na relační databázi bude třeba vytvořit schéma dle dříve zmíněných tabulek a~zvolení databázového systému.
Pro účely této práce byly zvoleny 3 následující databázové systémy: Microsoft SQL, PostgreSQL a~Oracle.
\pagebreak

\section{Microsoft SQL}
Microsoft SQL Server je relační databázový systém vyvinutý společností Microsoft.
Jedná se o~placený software, který je dostupný pro operační systémy Windows a~Linux.
SQL Server je dostupný ve více edicích, které se liší funkcemi a~cenou.
Nejčastěji používanou edicí je Standard Edition, která obsahuje všechny základní funkce.
SQL Server je vhodný pro velké a~střední podniky, které potřebují spolehlivý a~výkonný databázový systém.
\cite{microsoft_sql_server}

\section{PostgreSQL}
PostgreSQL je open-source relační databázový systém vyvinutý jako alternativa k~SQL Serveru.
Jedná se o~jeden z~open-source databázových systémů.
PostgreSQL je dostupný zdarma a~je k dispozici pro všechny hlavní operační systémy.
Díky své otevřené povaze je PostgreSQL oblíbený mezi vývojáři a~malými firmami.
\cite{postgresql}

\section{Oracle}
Oracle Database je relační databázový systém vyvinutý společností Oracle.
Jedná se o~placený software, který je dostupný pro operační systémy Windows, Linux a~Unix.
Oracle Database je dostupný ve více edicích, které se liší funkcemi a~cenou.
Nejčastěji používanou edicí je Standard Edition, která obsahuje všechny základní funkce.
Oracle Database je vhodný pro velké a~střední podniky, které potřebují spolehlivý a~výkonný databázový systém.
\cite{oracle_database}

\pagebreak
\section{Komunikace s~databází}
Všechny tři databázové systémy podporují komunikaci pomocí SQL jazyka.
SQL (Structured Query Language) je standardizovaný jazyk pro práci s~relačními databázemi,
který umožňuje vytváření, čtení, aktualizaci a~mazání dat.
Pro komunikaci s~databází je tedy třeba vytvořit SQL dotazy, které budou provádět operace nad daty.

Při implementaci je možné využít knihovny a~frameworky, které usnadňují práci se SQL dotazy, 
jako například JDBC pro Javu, psycopg2 pro Python nebo Active Record v~Ruby. 
Tyto nástroje nabízí přístup k~databázi, které usnadňují složitější operace.

Další možností je využití ORM (Object-Relational Mapping), který umožňuje mapování objektů na~tabulky v~databázi.
Mezi populární ORM nástroje patří Hibernate pro Javu, SQLAlchemy pro Python nebo Entity Framework pro .NET. 
Tyto nástroje umožňují programátorům pracovat s~databází více intuitivním způsobem, například pomocí objektů 
místo přímého psaní SQL dotazů. ORM také usnadňuje správu databázových transakcí a~poskytuje nástroje pro migrace, 
které zajišťují snadnou aktualizaci struktury databáze v~průběhu času.

Výsledkem tét práce bude aplikace, vytvořena v~jazyce Java, takže bude využita knihovna JDBC pro komunikaci 
s~databází a~ORM Hibernate pro mapování objektů na tabulky v~databázi.

\chapter{Konfigurační soubor}
\section{Co je to konfigurační soubor}
Konfigurační soubor je soubor, který obsahuje nastavení aplikace.
V konkrétním případě se jedná o~nastavení:
\begin{itemize}
    \item Databáze -- Connection string
    \item Zdroj -- Webová služba
    \item Výčet tabulek + mapování
    \item Výčet sloupců tabulek + mapování
    \item Plánovač -- Interval stahování
    \item Vytvoření databáze -- ANO/NE
\end{itemize}
Dále bude třeba vybrat vhodný formát pro konfigurační soubor.

\section{Formát}
Pro konfigurační soubor je možné zvolit několik formátů:
\begin{itemize}
    \item XML(XSD) -- Extensible Markup Language
    \item JSON -- JavaScript Object Notation
    \item YAML -- YAML Ain't Markup Language
\end{itemize}
Bude třeba víceúrovňový formát, který umožní snadné čtení a~zápis.
XML je zbytečně složité a~nepřehledné.
JSON je jednoduchý a~přehledný, ale nepodporuje komentáře.
YAML je jednoduchý a~přehledný, podporuje komentáře, ale je méně rozšířený.
Pro účely této práce byl zvolen formát JSON \cite{cisco_xml_json_yaml}.

\chapter{Technologie}
Některé technologie pro tuto práci již byly zmíněny (Java, JDBC, Hibernate, Quartz, atd.).
Pro účely této práce budou třeba ještě další technologie, které umožní tvorbu aplikace.

\section{Rest API}
REST API (Representational State Transfer Application Programming Interface) je architektura pro komunikaci mezi webovými službami,
která umožňuje přenos dat mezi klientem a~serverem \cite{rest_api}.
Tato služba bude hlavně využita při komunikaci se všemi službami, které budou potřeba pro tuto aplikaci.

\section{Spring Framework}
Spring Framework je open-source framework pro vývoj aplikací v~jazyce Java.
V tomto frameworku bude vytvořena aplikace, která bude vše spojovat dohromady.
Spring Framework obsahuje mnoho modulů, které usnadňují vývoj aplikací, 
jako například Spring Boot, Spring Data, Spring Security, atd.
Pro účely této práce bude využit modul Spring Boot, který umožňuje rychlé 
vytvoření aplikace s~minimální konfigurací.
\cite{spring_framework}

\pagebreak
\section{Plánovač}
V~popisu co bude potřebovat konfigurační soubor bylo zmíněno, že~bude třeba určit plánovač.
Co je to plánovač?
Plánovač je nástroj, který umožňuje spouštění úloh v~pravidelných intervalech.
Výhodou plánovače je, že umožňuje automatické spouštění úloh bez nutnosti manuálního zásahu.
Je zde několik možných plánovačů, které lze použít:
\begin{itemize}
    \item Cron -- Unix
    \item Task Scheduler -- Windows
    \item Quartz -- Java
    \item Apache Airflow -- Python
\end{itemize}
Všechny tyto plánovače umožňují spouštění úloh v~pravidelných intervalech.
Pro účely této práce byl zvolen plánovač Quartz, který je napsán v~jazyce Java a 
umožňuje spouštění úloh v~pravidelných intervalech.
Intervaly mohou být nastaveny v~cron notaci, což umožňuje velkou flexibilitu při plánování úloh
(každý den v~3:00, každý týden v~pondělí v~8:00, každý měsíc první den v~12:00 atd.).

\section{Docker}
Docker je open-source platforma pro vývoj, nasazení a~provoz aplikací.
Dále umožňuje vytváření kontejnerů, které obsahují všechny potřebné závislosti pro běh aplikace.
Proč je třeba docker?
Pro každou databázi je třeba vytvořit instanci, která bude obsahovat všechny potřebné tabulky, data 
a~klienta pro komunikaci a~práci s databází. Na~to se hodí docker, který umožňuje vytvoření kontejneru s~databází
a~klientem, který bude obsahovat všechny potřebné závislosti pro běh aplikace.
Dále zde může běžet i~plánovač, který bude spouštět úlohy v~pravidelných intervalech.
\cite{docker}


%---------------------------------------------------------------------------
\appendix
\chapter{První příloha}
\backmatter
\printbibliography
\setbackpageqrcode
\backpage
\end{document}