\chapter{Implementace aplikace}
Před začátkem popisu implementace aplikace je dobré říct, o co se aplikace bude snažit.
Cílovým účelem aplikace je stahování dat z~API RÚIAN a jejich následné zpracování.
Výsledkem bude projekce dat do databáze, která bude následně použita pro další zpracování.
Aplikace bude napsána v~programovacím jazyce Java a~použije framework Spring Boot.

\section{Projekce databáze}
Projekce databáze je způsob zobrazení dat z jedné databáze do druhé databáze.
Podle konfigurace se nastaví úroveň projekce a~podle toho se budou zobrazovat data.
Nastavení projekce bude dále popsáno v~sekci \ref{sec:konfigurace}.

\section{Inicializace aplikace}
Na začátku aplikace se provede její inicializace.
Důvod, proč byl vybrán právě framework Spring Boot, je funkčnost aplikace spočívající v~automatickém načtení všech komponent a~konfigurací.
Konkrétně se jedná o~moduly označené anotacemi @Component, @Service, @Configuration, @Entity, @Repository a~@Bean.

\subsection{Připojení k databázi}
O připojení k~databázi se stará třída \texttt{DatabaseSource}, která je zodpovědná za navázání spojení s databází.
Z konfiguračního souboru se načtou potřebné informace pro připojení k databázi.
Konkrétně se jedná o~tyto informace:
\begin{itemize}
    \item \texttt{type} -- typ databáze (např. \texttt{postgresql}, \texttt{mssql}, \texttt{oracle})
    \item \texttt{url} -- adresa databáze (např. \texttt{localhost:5432} pro PostgreSQL)
    \item \texttt{dbname} -- název databáze (např. \texttt{ruian})
    \item \texttt{username} -- uživatelské jméno pro připojení k databázi
    \item \texttt{password} -- heslo pro připojení k databázi
\end{itemize}

Podle těchto informací se vytvoří připojení k databázi, neboli \textbf{DataSource}.
\texttt{DataSource} slouží jako zdroj dat pro JPA a~je zodpovědný za správu spojení s databází.
Tento zdroj bude dále upraven v jiném modulu.
Základem \texttt{DataSource} je nastavení prvotních parametrů připojení.
Vytváří se tedy výsledný connection string, který se modifikuje podle typu databáze.
K URL se připojí název databáze, uživatelské jméno, heslo, a~v případě MSSQL se přidává i certifikát pro zabezpečené připojení.

V dalším modulu se pro \texttt{DataSource} donastavují další parametry pro přístup k databázi a~přenos dat.
Jako první se inicializují DTO objekty, repozitáře a~třídy typu \texttt{Service} pro JPA.
Tyto objekty se inicializují pomocí anotace \texttt{@Autowired}, která je zodpovědná za injektování závislostí.
Dále se nastaví dialekt pro správnou syntaxi SQL příkazů podle dané databáze.

\subsection{Načtení konfigurace}
\label{sec:konfigurace}
Zatímco \texttt{DatabaseSource} se stará pouze o~nastavení a připojení k databázi,
třída \texttt{AppConfig} se stará o~načtení zbytku informací.
Konfigurace se načítá následovně:
\begin{enumerate}
    \item Konfigurace úkolů pro \texttt{Quartz Scheduler}. 
    Konkrétně se načítá čas ve formátu cron pro spouštění přidávání přírůstkových dat.
    Dále se načítá informace, zda přeskočit základní inicializaci pro hlavní územní prvky a kraje.
    
    \item Seznam krajů s příslušnými kódy. Každý řádek v konfiguračním souboru obsahuje kraj s jeho kódem.
    Pokud je v konfiguraci pro \texttt{Quartz Scheduler} nastaveno přeskočení inicializace krajů nebo je seznam prázdný,
    tento krok je zcela přeskočen.
    
    \item Dodatečná nastavení. Především se jedná o~nastavení pro ignorování geometrických dat.
    Toto je implementováno z~důvodu, že některé databáze nepodporují geometrické datové typy.
    Dále je zde nastavení velikosti jednotlivých commitů do databáze, což je prvek optimalizace výkonu.
    
    \item Nastavení zpracování jednotlivých tabulek. Hlavním parametrem je způsob zpracování tabulek, který může být
    \texttt{all} nebo \texttt{selected}. Pokud je nastavení \texttt{all}, budou zpracovány všechny tabulky bez ohledu na další konfiguraci.
    
    \begin{itemize}
        \item \textbf{all} -- všechny tabulky budou zpracovány bez ohledu na konfiguraci jednotlivých tabulek.
        \item \textbf{selected} -- budou zpracovány pouze tabulky, které jsou výslovně uvedeny v konfiguraci.
    \end{itemize}
    
    \item Konfigurace jednotlivých tabulek. Pokud je nastaven režim \texttt{selected}, zpracovávají se jen specifikované tabulky.
    Každá tabulka může mít své vlastní nastavení:

    \begin{lstlisting}[language=json, caption=Konfigurace tabulek, label=lst:konfTabulek]
"<table_name>": {
    "howToProcess": "all | exclude | include",
    "columns": ["<column_name>", ..., "<column_name>"]
}
    \end{lstlisting}

    Možnosti zpracování tabulky:
    \begin{itemize}
        \item \textbf{all} -- všechny sloupce budou zpracovány bez ohledu na konfiguraci.
        \item \textbf{exclude} -- vybrané sloupce budou ignorovány, ostatní budou zpracovány.
        \item \textbf{include} -- vybrané sloupce budou zpracovány, ostatní budou ignorovány.
    \end{itemize}

    Sloupce jsou definovány jako pole řetězců s názvy sloupců.
\end{enumerate}

Možné chyby při načítání konfigurace:
\begin{itemize}
    \item Pokud je u tabulky nastaveno \texttt{exclude} nebo \texttt{include}, ale nejsou uvedeny žádné sloupce \textrightarrow{} nevalidní konfigurace.
    \item Sloupce nebo tabulky, které neexistují v databázi \textrightarrow{} budou přeskočeny a~nebudou zpracovány.
    \item Seznam krajů je povinný atribut, i pokud je prázdný. Pokud je prázdný, budou zpracovány pouze základní územní prvky.
\end{itemize}

\newpage

\section{Quartz Scheduler}
\subsection{Jobs}
\subsection{Triggers}

\section{Stahování dat}
\subsection{Stahování dat z~API}

\section{Zpracování dat}
\subsection{Dto objekty}
\subsection{Repositáře}
\subsection{Service třídy}
\subsection{Parsing dat}
\subsection{Validace dat}
\subsection{Ukládání dat do databáze}