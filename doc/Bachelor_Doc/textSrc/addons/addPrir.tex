\section*{Co je to Ruian Puller?}
Tato aplikace stahuje data z~\url{https://vdp.cuzk.cz/vdp/ruian/vymennyformat} a~ukládá je do~SQL databáze.
Podporované databáze:
\begin{itemize}
  \item \textbf{Microsoft SQL}
  \item \textbf{PostgreSQL}
  \item \textbf{Oracle}
\end{itemize}

Data jsou stahována ve formátu XML a~převáděna do~DTO (Data Transfer Object), které jsou následně ukládány do~databáze pomocí JPA (Java Persistence API).\\
Aplikace je napsána v~Javě~21 a~používá Spring Boot pro snadné nastavení a~konfiguraci. Všechny závislosti jsou spravovány pomocí Maven.\\
Aplikace dokáže stahovat a~ukládat standardní datové sady:
\begin{itemize}
  \item Stát až ZSJ (základní)
  \item Všechny kraje a~obce (základní)
  \item Přírůstková data (základní)
\end{itemize}

\section*{Před prvním spuštěním}
Je třeba mít zprovozněné SQL databáze pro ukládání dat. Databáze může běžet podle vlastních nastavení, nebo lze použít variantu s~Dockerem.
V~adresáři \texttt{database} se nachází podadresáře pro každou z~podporovaných databází. Každý podadresář obsahuje \texttt{docker-compose.yml} a~\texttt{init.sql},
které vytvoří databázi a~tabulky pro ukládání dat.
Oracle databáze neobsahuje \texttt{docker-compose.yml}, protože byla vytvořena mimo Docker.

\subsection*{Docker MS SQL a~PostgreSQL}
Pro spuštění v~Dockeru je zde adresář \texttt{db}, kde jsou dva podadresáře: PostgreSQL a~MS SQL. V~každém je příslušný \texttt{docker-compose.yml} a~\texttt{init.sql}.

Pro vytvoření instancí spusťte v~příslušném adresáři:
\begin{lstlisting}[language=bash]
docker-compose up -d
\end{lstlisting}

Pro případné vyčištění:
\begin{lstlisting}[language=bash]
docker-compose down -v
\end{lstlisting}

\subsection*{Oracle}
Oracle databáze je vytvořena mimo Docker. Je třeba mít nainstalovanou Oracle Database 19c nebo vyšší. Pro vytvoření instance použijte \texttt{init.sql}.
Chování databáze je následně stejné jako u~předchozích dvou.

\section*{Aplikace}
Aplikace se nachází v~adresáři \texttt{app}, kde je zdrojový kód a~Maven projekt. Je napsána v~Javě~21 a~používá Spring Boot. Závislosti jsou spravovány pomocí Maven.
Projekt je nastaven na~JDK~21 a~Maven~3.8.6 a~je spustitelný v~IntelliJ IDEA nebo Eclipse.

\section*{Konfigurace}
Konfigurační soubor \texttt{config.json} obsahuje nastavení připojení k~databázi, seznam tabulek a~další volby. Nachází se v~adresáři \texttt{src/main/resources}.

\newpage

\begin{lstlisting}[language=json, caption={Příklad konfiguračního souboru}]
{
  "database": {
    "type": "<databse_type>",
    "url": "<connection_string>",
    "dbname": "<database_name>",
    "username": "<username>",
    "password": "<password>"
  },
  "quartz": {
    "cron": "0 0 2 * * ?",
    "skipInitialRunStat": false,
    "skipInitialRunRegion": true
  },
  "vuscCodes": {
    "<kod>": "<kraj_nazev>"
  },
  "additionalOptions": {
    "includeGeometry": true,
    "commitSize": 1000
  },
  "dataToProcess": {
    "howToProcess": "<all/selected>",
    "tables": {
      "<tablename>": {
        "howToProcess": "<all,include,exclude>",
        "columns": ["<column_name>", "<column_name>", ...]
      }
    }
  }
}
\end{lstlisting}

\newpage

\subsection*{Popis jednotlivých částí}
\begin{itemize}
  \item \textbf{database} -- nastavení připojení k~databázi:
  \begin{itemize}
    \item \texttt{type}: typ databáze (postgresql, mssql, oracle)
    \item \texttt{url}: connection string
    \item \texttt{dbname}, \texttt{username}, \texttt{password}
  \end{itemize}

  \item \textbf{quartz} -- plánovač stahování dat:
  \begin{itemize}
    \item \texttt{cron}: výraz pro plánovač
    \item \texttt{skipInitialRunStat} -- přeskočit inicializaci Stát až ZSJ (true/false)
    \item \texttt{skipInitialRunRegion} -- přeskočit inicializaci krajů (true/false)
  \end{itemize}

  \item \textbf{vuscCodes} -- kódy krajů a~obcí:
  \begin{longtable}{|c|c|}
    \caption{Seznam krajů a~jejich kódů} \\
    \hline
    \textbf{Kód} & \textbf{Kraj} \\
    \hline
    19 & Hlavní město Praha \\
    27 & Jihočeský kraj \\
    35 & Jihomoravský kraj \\
    43 & Karlovarský kraj \\
    51 & Kraj Vysočina \\
    60 & Královéhradecký kraj \\
    78 & Liberecký kraj \\
    86 & Moravskoslezský kraj \\
    94 & Olomoucký kraj \\
    108 & Pardubický kraj \\
    116 & Plzeňský kraj \\
    124 & Středočeský kraj \\
    132 & Ústecký kraj \\
    141 & Zlínský kraj \\
    \hline
  \end{longtable}

  \newpage

  \item \textbf{additionalOptions} -- další volby:
  \begin{itemize}
    \item \texttt{includeGeometry} -- zahrnout geometrii (true/false)
    \item \texttt{commitSize} -- velikost dávky pro commit (default 1000)
  \end{itemize}

  \item \textbf{dataToProcess} -- nastavení pro zpracování dat:
  \begin{itemize}
    \item \texttt{howToProcess} -- jak zpracovat data (all/selected)
    \item \texttt{tables} -- nastavení pro jednotlivé tabulky:
    \begin{itemize}
      \item \texttt{howToProcess} -- jak zpracovat tabulku (all/include/exclude)
      \item \texttt{columns} -- seznam sloupců pro zpracování
    \end{itemize}
  \end{itemize}
\end{itemize}

\begin{center}
  \begin{longtable}{|>{\raggedright\arraybackslash}p{4cm}|>{\raggedright\arraybackslash}p{9cm}|}
    \caption{Seznam tabulek a sloupců} \\

    \hline
    \textbf{Tabulka} & \textbf{Sloupce} \\
    \hline
    \endfirsthead
    
    \hline
    \textbf{Tabulka} & \textbf{Sloupce} \\
    \hline
    \endhead
  
    stat & nazev, nespravny, platiod, platido, idtransakce, globalniidnavrhuzmeny, nutslau, geometriedefbod, geometriegenhranice, geometrieorihranice, nespravneudaje, datumvzniku \\
    \hline
    regionSoudrznosti & nazev, nespravny, stat, platiod, platido, idtransakce, globalniidnavrhuzmeny, nutslau, geometriedefbod, geometriegenhranice, geometrieorihranice, nespravneudaje, datumvzniku \\
    \hline
    vusc & nazev, nespravny, regionsoudrznosti, platiod, platido, idtransakce, globalniidnavrhuzmeny, nutslau, geometriedefbod, geometriegenhranice, geometrieorihranice, nespravneudaje, datumvzniku \\
    \hline
    okres & nazev, nespravny, kraj, vusc, platiod, platido, idtransakce, globalniidnavrhuzmeny, nutslau, geometriedefbod, geometriegenhranice, geometrieorihranice, nespravneudaje, datumvzniku \\
    \hline
    orp & nazev, nespravny, spravniobeckod, vusc, okres, platiod, platido, idtransakce, globalniidnavrhuzmeny, geometriedefbod, geometriegenhranice, geometrieorihranice, nespravneudaje, datumvzniku \\
    \hline
    pou & nazev, nespravny, spravniobeckod, orp, platiod, platido, idtransakce, globalniidnavrhuzmeny, geometriedefbod, geometriegenhranice, geometrieorihranice, nespravneudaje, datumvzniku \\
    \hline
    obec & nazev, nespravny, statuskod, okres, pou, platiod, platido, idtransakce, globalniidnavrhuzmeny, mluvnickecharakteristiky, vlajkatext, vlajkaobrazek, znaktext, znakobrazek, clenenismrozsahkod, clenenismtypkod, nutslau, geometriedefbod, geometriegenhranice, geometrieorihranice, nespravneudaje, datumvzniku \\
    \hline
    castObce & nazev, nespravny, obec, platiod, platido, idtransakce, globalniidnavrhuzmeny, mluvnickecharakteristiky, geometriedefbod, nespravneudaje, datumvzniku \\
    \hline
    mop & nazev, nespravny, obec, platiod, platido, idtransakce, globalniidnavrhuzmeny, geometriedefbod, geometrieorihranice, nespravneudaje, datumvzniku \\
    \hline
    spravniObvod & nazev, nespravny, spravnimomckod, obec, platiod, platido, idtransakce, globalniidnavrhuzmeny, geometriedefbod, geometrieorihranice, nespravneudaje, datumvzniku \\
    \hline
    momc & nazev, nespravny, spravniobvod, mop, obec, spravniobvod, platiod, platido, idtransakce, globalniidnavrhuzmeny, vlajkatext, vlajkaobrazek, znaktext, znakobrazek, mluvnickecharakteristiky, geometriedefbod, geometrieorihranice, nespravneudaje, datumvzniku \\
    \hline
    katastralniUzemi & nazev, nespravny, existujedigitalnimapa, obec, platiod, platido, idtransakce, globalniidnavrhuzmeny, rizeniid, mluvnickecharakteristiky, geometriedefbod, geometriegenhranice, nespravneudaje, datumvzniku \\
    \hline
    parcela & nespravny, kmenovecislo, pododdelenicisla, vymeraparcely, zpusobyvyuzitipozemku, druhcislovanikod, druhpozemkukod, katastralniuzemi, platiod, platido, idtransakce, rizeniid, bonitovanedily, zpusobyochranypozemku, geometriedefbod, geometrieorihranice, nespravneudaje \\
    \hline
    ulice & nazev, nespravny, obec, platiod, platido, idtransakce, globalniidnavrhuzmeny, geometriedefbod, geometriedefcara, nespravneudaje \\
    \hline
    stavebniObjekt & cislodomovni, identifikacniparcela, typstavebnihoobjektukod, castobce, momc, platiod, platido, idtransakce, globalniidnavrhuzmeny, isknbudovaid, dokonceni, druhkonstrukcekod, obestavenyprostor, pocetbytu, pocetpodlazi, podlahovaplocha, pripojenikanalizacekod, pripojeniplynkod, pripojenivodovodkod, vybavenivytahemkod, zastavenaplocha, zpusobvytapenikod, zpusobyochrany, detailnitea, geometriedefbod, geometrieorihranice, nespravneudaje \\
    \hline
    adresniMisto & nespravny, cislodomovni, cisloorientacni, cisloorientacnipismeno, psc, stavebniobjekt, ulice, vokod, platiod, platido, idtransakce, globalniidnavrhuzmeny, geometriedefbod, nespravneudaje \\
    \hline
    zsj & nazev, nespravny, katastralniuzemi, platiod, platido, idtransakce, globalniidnavrhuzmeny, mluvnickecharakteristiky, vymera, charakterzsjkod, geometriedefbod, geometriegenhranice, geometrieorihranice, nespravneudaje, datumvzniku \\
    \hline
    vo & platiod, platido, idtransakce, globalniidnavrhuzmeny, geometriedefbod, geometriegenhranice, geometrieorihranice, nespravneudaje, cislo, nespravny, obec, momc, poznamka \\
    \hline
    zaniklyPrvek & typprvkukod, idtransakce \\
    \hline
    
  \end{longtable}
\end{center}

\textit{Poznámky:}
\begin{enumerate}
  \item Quartz Scheduler obsahuje dvě boolean nastavení pro přeskočení kroků inicializace.
  \item Všechny tabulky kromě \texttt{vo} byly testovány. Je tedy možné, že při parsování dojde k~chybě.
\end{enumerate}

\section*{Průběh aplikace}
\begin{enumerate}
  \item Aplikace načte nastavení z~\texttt{config.json}.
  \item Pokud je \texttt{skipInitialRunStat = false}, stáhne se Stát až ZSJ.
  \item Pokud je \texttt{skipInitialRunRegion = false}, stáhnou se vybrané kraje.
  \item Následně se stahují přírůstková data.
  \item Aplikace čeká na~další běh dle Quartz Scheduleru.
\end{enumerate}

Aplikace nepřepisuje data, pouze přidává nové nebo aktualizuje existující záznamy.
Pokud nastane chyba typu Foreign Key, problémový objekt se přeskočí a~pokračuje se dál.