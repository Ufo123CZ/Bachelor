Některé technologie pro tuto práci již byly zmíněny (Java, JDBC, Hibernate atd.) v~sekci \ref{sec:komunikaceDB}.  
Pro účely této práce budou potřeba ještě další technologie, které umožní tvorbu aplikace.

\section{REST API}
REST API (Representational State Transfer Application Programming Interface) je architektonický  
styl pro návrh webových služeb, který umožňuje snadnou a~efektivní komunikaci mezi klientem a~serverem.  
Tento přístup se stal jedním z~nejrozšířenějších způsobů integrace aplikací díky své jednoduchosti,  
flexibilitě a~nezávislosti na platformě. REST API využívá standardní metody protokolu HTTP, jako jsou  
GET, POST, PUT a~DELETE, k~provádění různých operací s~daty.

REST API poskytuje ideální prostředí pro implementaci přenosu dat díky své flexibilitě a~schopnosti  
pracovat s~různými datovými zdroji. V~této práci bude kladen důraz na robustnost a~spolehlivost API,  
což zahrnuje zpracování chybových stavů, zabezpečení komunikace (například prostřednictvím HTTPS)  
a~optimalizaci výkonu.  
\cite{rest_api}

\section{Spring Framework}
Spring Framework je open-source framework pro vývoj aplikací v~jazyce Java.  
V~tomto frameworku bude vytvořena aplikace, která bude vše spojovat dohromady.  
Spring Framework obsahuje mnoho modulů, které usnadňují vývoj aplikací,  
jako například Spring Boot, Spring Data, Spring Security atd.  
Pro účely této práce bude využit modul Spring Boot, který umožňuje rychlé  
vytvoření aplikace s~minimální konfigurací.  
\cite{spring_framework}

\section{Maven}
Maven je nástroj pro správu projektů v~jazyce Java, který usnadňuje správu závislostí,
kompilaci a~nasazení aplikací. Jedná se o jednoduchý a~efektivní nástroj, který umožňuje
správu projektů pomocí XML konfiguračního souboru.
Konkrétně se jedná o~soubor \texttt{pom.xml}, který obsahuje informace o~projektu, 
jako jsou závislosti, pluginy a~další nastavení.
\cite{maven}

\section{Plánovač}
V~popisu, co bude obsahovat konfigurační soubor, bylo zmíněno, že~bude třeba zvolit plánovač.  
Plánovač je nástroj, který umožňuje spouštění úloh v~pravidelných intervalech.  
Výhodou plánovače je, že umožňuje automatické spouštění úloh bez nutnosti manuálního zásahu.  

Existuje několik možných plánovačů, které lze použít:
\begin{itemize}
    \item \textbf{Cron} -- Unix,
    \item \textbf{Task Scheduler} -- Windows,
    \item \textbf{Quartz} -- Java,
    \item \textbf{Apache Airflow} -- Python,
\end{itemize}

Všechny tyto plánovače umožňují spouštění úloh v~pravidelných intervalech.  
Pro účely této práce byl zvolen plánovač Quartz, který je napsán v~jazyce Java a  
je podporován Spring Frameworkem.  

Intervaly mohou být nastaveny v~cron notaci, což umožňuje velkou flexibilitu při plánování úloh  
(například každý den ve 3:00, každý týden v~pondělí v~8:00, každý měsíc první den ve 12:00 atd.).

\section{Docker}
Docker je open-source platforma pro vývoj, nasazení a~provoz aplikací.  
Umožňuje vytváření kontejnerů, které obsahují všechny potřebné závislosti pro běh aplikace.  

Pro každou databázi je třeba vytvořit instanci, která bude obsahovat potřebné tabulky, data  
a~klienta pro komunikaci a~práci s~databází. Toto je úloha jako stvořená pro Docker,  
který umožňuje vytvoření kontejneru s~databází, který bude obsahovat veškeré potřebné závislosti  
pro běh aplikace.  
\cite{docker}

\newpage

\section{Grafický klient pro správu databáze}
Klient pro komunikaci s~databází je nástroj, který umožňuje připojení k~databázi a~provádění dotazů.  
Je třeba vybrat klienta, který bude podporovat náhled do všech databází použitých v~této práci.

Možnosti jsou:
\begin{itemize}
    \item \textbf{DBeaver} -- open-source nástroj pro správu databází, který podporuje mnoho různých databází.
    \item \textbf{SQL Developer} -- nástroj pro správu databází Oracle.
    \item \textbf{Adminer} -- open-source nástroj pro správu databází, který může zároveň běžet v~Dockeru.
\end{itemize}

Všechny tyto nástroje umožňují připojení k~databázi a~provádění dotazů.  
Pro účely této práce byl zvolen DBeaver, který podporuje širokou škálu databází a~edice Community Edition je open-source.  
Zároveň umožňuje export dat z~databáze do různých formátů (CSV, Excel, JSON atd.).  
\cite{dbeaver}

\section{Log4j}
Log4j je open-source knihovna pro logování v~jazyce Java.
Umožňuje snadné a~efektivní logování událostí v~aplikaci.
Je možné nastavit různé úrovně logování (DEBUG, INFO, WARN, ERROR, FATAL) a~také různé výstupy 
(soubor, konzole, databáze atd.).
Tato technologie bude používána v průběhu vývoje aplikace pro logování událostí.
\cite{log4j}