Databázový systém je softwarový nástroj, který slouží k~efektivnímu ukládání, organizaci a~vyhledávání dat. 
Díky své strukturované povaze umožňuje správu velkých objemů dat a~poskytuje funkcionality pro 
zajištění konzistence, bezpečnosti a~rychlého přístupu k~uloženým informacím.

Databázový systém lze obecně rozdělit do~dvou hlavních kategorií: \textbf{relační} a~\textbf{objektové databázový systém}. 
Každý z~těchto přístupů má své specifické vlastnosti a~je vhodný pro odlišné typy aplikací.

\subsubsection*{Relační databázový systém}
\begin{itemize}[itemsep=-1pt]
    \item Data jsou uložena v~tabulkách.
    \item Každý řádek tabulky obsahuje jeden záznam.
    \item Každý sloupec tabulky obsahuje jeden atribut.
    \item Vztahy mezi tabulkami jsou definovány klíči.
    \item Využití jazyka SQL.
\end{itemize}
Relační databázový systém jsou vhodné pro strukturovaná data, která mají pevnou strukturu.
Jedná se o~nejčastěji používaný typ databázových systémů.

\subsubsection*{Objektové databázový systém}
\begin{itemize}[itemsep=-1pt]
    \item Data jsou uložena jako objekty.
    \item Každý objekt obsahuje atributy a~metody.
    \item Vztahy mezi objekty jsou definovány referencemi.
    \item Využití objektově orientovaného jazyka.
\end{itemize}
Objektové databázové systémy jsou vhodné pro nestrukturovaná data, která mají složitou strukturu.
Jedná se o~novější typ databázových systémů, který je vhodný pro moderní aplikace.

Vzhledem k~zadání, kde je přímo specifikováno, které databázové systémy budou použity,
není třeba vybírat mezi těmito dvěma systémy. Všechny tři databáze, které budou popsány,
jsou relační databázové systémy. Tyto systémy však mají mezi sebou rozdíly v~použití, funkcích a~možnostech.

Databáze bude potřebovat některé dodatečné funkce, které jsou nezbytné pro práci s~daty.
Mezi tyto funkce patří:
\begin{itemize}
    \item Zpracování geometrických dat.
    \item Podpora JSON.
    \item Podpora vhodných datových typů (čas a~datum, čísla, text).
\end{itemize}

\section{Microsoft SQL Server}
Microsoft SQL Server je relační databázový systém, který vyvinula společnost 
Microsoft a~který se stal jedním z~předních nástrojů pro ukládání, správu a~analýzu dat. 
SQL Server je robustní a~výkonný systém, který nabízí širokou škálu funkcí 
a~možností přizpůsobení pro různé typy aplikací. Díky své dlouhodobé podpoře a~integraci 
s~dalšími produkty společnosti Microsoft, jako je Azure nebo Power BI, je SQL Server 
oblíbenou volbou pro velké a~střední podniky.

Edice SQL Serveru zahrnují Standard, Enterprise a~Express, 
které se liší funkcemi a~cenou. Standard Edition je nejčastěji používanou edicí, 
která obsahuje všechny základní funkce a~je vhodná pro většinu aplikací.

SQL Server byl původně navržen výhradně pro Windows, ale od verze SQL Server 2017 je dostupný 
také pro operační systém Linux. Tato multiplatformní podpora zvyšuje jeho použitelnost 
v~různých IT prostředích.
\cite{microsoft_sql_server}

Pro tuto práci bude využita edice SQL Server 2017 Standard, která je dostupná pro Windows a~Linux.
Tato edice podporuje vhodné datové typy pro zpracování geometrických dat.

\section{PostgreSQL}
PostgreSQL je open source relační databázový systém, který je známý svou spolehlivostí, výkonem 
a~rozšiřitelností. Původně byl vyvinut jako alternativa k~proprietárním řešením, jako je SQL Server, 
a~dnes se řadí mezi nejpokročilejší relační databázové systémy na trhu. Díky své otevřené povaze 
a~aktivní komunitě uživatelů a~vývojářů se stal oblíbenou volbou nejen mezi malými firmami, 
ale také ve středních a~velkých organizacích.

PostgreSQL je dostupný zdarma a~podporuje všechny hlavní operační systémy, včetně Windows, Linux a~macOS. 
Tato multiplatformní dostupnost umožňuje snadnou integraci PostgreSQL do~různých vývojových prostředí.
\cite{postgresql}

PostgreSQL také podporuje jak formát JSON, tak všechny potřebné datové typy kromě geometrických dat.
Pro ukládání geometrických dat je třeba použít \textit{PostGIS}, což je rozšíření pro PostgreSQL, které přidává
podporu pro prostorové a~geometrické datové typy.

\section{Oracle Database}
Oracle Database, vyvinutá společností Oracle Corporation, patří mezi přední relační 
databázové systémy na světě. Tento systém je známý svou robustností, vysokým výkonem 
a~schopností zvládat kritické podnikové aplikace a~rozsáhlé datové sady. Oracle Database 
je navržena tak, aby poskytovala spolehlivé a~efektivní řešení pro ukládání, správu 
a~analýzu dat ve velkých i~středních organizacích.

Edice Oracle Database zahrnují Standard Edition, Enterprise Edition a~Express Edition,
které se liší funkcemi a~cenou. Enterprise Edition je nejkomplexnější edicí, která
obsahuje všechny pokročilé funkce a~je vhodná pro velké podniky s~náročnými požadavky.

Oracle Database je kompatibilní s~většinou hlavních operačních systémů, včetně Windows, 
Linux a~Unix. Díky tomu může být nasazena v~různorodých IT prostředích podle požadavků organizace.
\cite{oracle_database}

Pro tuto práci bude využita edice Oracle Express Edition, která je dostupná zdarma a~je určena pro vývoj a~testování.
Tato edice podporuje všechny potřebné datové typy a~také JSON. Ovšem není zde možnost 
uložení geometrických dat. Ta jsou obsažena v~\texttt{SDO\_GEOMETRY}, což je rozšíření pro Oracle Database.

\section{Komunikace s~využitím SQL}
Všechny tři databázové systémy podporují dotazovací jazyka SQL (Structured Query Language).
SQL je standardizovaný jazyk pro práci s~relačními databázovými systémy, které umožňuje vytváření, čtení, aktualizaci a~mazání dat.
Pro komunikaci s~databází je tedy třeba vytvořit SQL dotazy, které budou provádět operace nad daty.
O~tuto komunikaci se stará aplikační vrstva, která zajišťuje připojení k~databázi a~následné zpracování a~odeslání SQL dotazů.

Vytvářet dotazy, pro ukládání různých dat bude velmi náročné a~neefektivní.
Ve výsledné aplikaci bude použito ORM (Object-Relational Mapping), které umožňuje práci s~daty pomocí objektů.
ORM je technika, která mapuje objekty v~programovacím jazyce na tabulky v~databázi a~naopak.
Konkrétně bude použito \textit{Hibernate}, což je open-source ORM framework pro jazyk Java.

\section{JDBC}
Jedná se o~Java API pro přístup k~relačním databázovým systémům.
JDBC (Java Database Connectivity) poskytuje standardní rozhraní pro připojení k~databázím,
Bude použito pro připojení k~databázi a~provádění SQL dotazů.
Využívá různé ovladače pro různé databázové systémy, takže je možné připojit se k~jakékoli databázi,
která podporuje JDBC. 