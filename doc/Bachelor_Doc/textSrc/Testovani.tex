\section{Rychlostní porovnání databázových systémů}
Tento test byl proveden na stejném konfiguračním souboru pro všechny databázové systémy.
Testovací konfigurace je nastavena na provedení stažení, zpracování a~uložení Základní datové sady Stát~až~ZSJ. 
Dále bylo nastavené, že se zpracují všechny tabulky uvedené v~již zmíněné datové sadě.
Geometrie bude zpracována a~velikost jednotlivých commitů bude 2000.
Vzhled konfiguračního souboru je uveden v~příkladu \ref{lst:konfigTest} a~bude nakonfigurován pro PostgreSQL.
Testování probíhalo na stejném stroji, který měl nainstalované všechny databázové systémy.

Nad všemi databázovými systémy probíhaly testy \(3\times\), aby se eliminovaly chyby způsobené jinými procesy na serveru.
Každý test byl proveden na prázdné databázi.
Měření začalo, když byla data připravena ke čtení z důvodu eliminace stahování dat z~internetu.
Konkrétně když se vypsala zpráva \uv{Data proccesing started.} a~následně skončilo, když se 
vypsala zpráva \uv{Data proccesing finished.}.
Během tohoto testu byl také vyfiltrováno 1 ZSJ z důvodu neexistence cizího klíče v~tabulce Katastrální území.
Testování proběhlo s daty \url{https://vdp.cuzk.gov.cz/vymenny_format/soucasna/20250331_ST_UZSZ.xml.zip}
% TODO: make table for all databases and its times

\small
\begin{lstlisting}[language=json, caption={Konfigurační soubor pro test rychlosti}, label={lst:konfigTest}]
    {
        "database": {
          "type": "postgresql",
          "url": "jdbc:postgresql://localhost:5432",
          "dbname": "ruian",
          "username": "postgres",
          "password": "123"
        },
        "quartz": {
          "cron": "0 0 2 * * ?",
          "skipInitialRunStat": false,
          "skipInitialRunRegion": true
        },
        "vuscCodes": {},
        "additionalOptions": {
          "includeGeometry": true,
          "commitSize": 2000
        },
        "dataToProcess": {
          "howToProcess": "all"
        }
      }    
\end{lstlisting}
\normalsize

\section{Porovnání zdrojových a výsledných dat}
