\section{Rychlostní porovnání databázových systémů}
Tento test byl proveden na stejném konfiguračním souboru pro všechny databázové systémy.
Testovací konfigurace je nastavena na provedení stažení, zpracování a~uložení Základní datové sady Stát~až~ZSJ.
Tato datová sada byla vybrána z důvodu neměnnosti. 
Data v této datové sadě se mění jen velmi zřídka a~je možné je stáhnout.
Dále bylo nastavené, že se zpracují všechny tabulky uvedené v~již zmíněné datové sadě.
Geometrie bude zpracována a~velikost jednotlivých commitů bude 2000.
Vzhled konfiguračního souboru je uveden v~příkladu \ref{lst:konfigTest} a~bude konfigurován pro PostgreSQL.
Testování probíhalo na stejném stroji, který měl instalované všechny databázové systémy.

Nad všemi databázovými systémy probíhaly testy \(3\times\), aby se eliminovaly chyby způsobené jinými procesy na serveru.
Každý test byl proveden na prázdné databázi.
Měření začalo, když byla data připravena ke čtení z důvodu eliminace stahování dat z~internetu.
Konkrétně když se vypsala zpráva \uv{Data proccesing started.} a~následně skončilo, když se 
vypsala zpráva \uv{Data proccesing finished.}.
Během tohoto testu byl také vyfiltrováno 1 ZSJ z důvodu neexistence cizího klíče v~tabulce Katastrální území.
Testování proběhlo s daty \url{https://vdp.cuzk.gov.cz/vymenny_format/soucasna/20250331_ST_UZSZ.xml.zip}
Výsledky jsou uvedeny v~tabulce \ref{tab:test1} a mají formát HH:MM:SS.

\begin{table}[!h]
  \centering
  \caption{Časy testů pro používané databáze}
  \label{tab:test1}
  \begin{tabular}{|c|c|c|c|}
  \hline
                  & \textbf{PostgreSQL} & \textbf{MS SQL} & \textbf{Oracle} \\ \hline
  \textbf{test 1} & 0:35:59             & 0:22:46         & 1:04:19         \\ \hline
  \textbf{test 2} & 0:37:03             & 0:22:26         & 1:05:30         \\ \hline
  \textbf{test 3} & 0:35:53             & 0:21:26         & 1:04:57         \\ \hline
  \textbf{průměr} & 0:36:18             & 0:22:13         & 1:04:55         \\ \hline
  \end{tabular}
\end{table}

Jak je vidět v tabulce \ref{tab:test1}, ukázalo se že MS SQL je nejrychlejší databázový systém pro zpracování datové sady.
PostgreSQL je o~přesně 14 minut pomalejší než MS SQL a~Oracle je o~přesně 28 minut pomalejší než MS SQL.
Je ale možné že Oracle je pomalejší z důvodu že se jedná o~Express verzi, která je omezena na 2GB RAM a~1 CPU.
Dále je možné že rychlost byla omezena prostředím. Zatím co PostgreSQL a MS SQL běžely v Docker kontejneru, 
Oracle běžel přímo na hostitelském systému.

\begin{lstlisting}[language=json, caption={Konfigurační soubor pro test rychlosti}, label={lst:konfigTest}]
    {
        "database": {
          "type": "postgresql",
          "url": "jdbc:postgresql://localhost:5432",
          "dbname": "ruian",
          "username": "postgres",
          "password": "123"
        },
        "quartz": {
          "cron": "0 0 2 * * ?",
          "skipInitialRunStat": false,
          "skipInitialRunRegion": true
        },
        "vuscCodes": {},
        "additionalOptions": {
          "includeGeometry": true,
          "commitSize": 2000
        },
        "dataToProcess": {
          "howToProcess": "all"
        }
      }    
\end{lstlisting}


\section{Testování kompletního importu}
