\section{Inicializace aplikace}
Na začátku běhu aplikace se provede její inicializace.  
Důvodem, proč byl zvolen framework Spring Boot, je schopnost aplikace automaticky 
načítat všechny komponenty a~konfigurace.  
Konkrétně se jedná o~moduly označené anotacemi \texttt{@Component}, \texttt{@Service}, 
\texttt{@Configuration}, \texttt{@Entity}, \texttt{@Repository} a~\texttt{@Bean}.

\section{Mapování dat}
\label{sec:mapovaniDat}
Pro mapování dat mezi databází a~aplikací se používá \texttt{Hibernate ORM}.
\texttt{Hibernate} je objektově-relační mapovací framework, který umožňuje práci s~databází pomocí objektů.
Pro mapování byly použity \textbf{Dto objekty}, které reprezentují jednotlivé tabulky v~databázi.
Následně je zde Repository interface, které dědí z~\texttt{JpaRepository}.
A nakonec je zde Service třída, která obsahuje logiku pro práci s~databází a~komunikuje s~Repository.

\subsection{Dto objekty}
Dto objekty nebo také \textbf{Data Transfer Object} jsou objekty, které slouží k~přenášení dat mezi vrstvami aplikace.
Pro jejich označení se používá anotace \texttt{@Entity}, která určuje, že se jedná o~entitu mapovanou na tabulku v~databázi.
Dále mají anotaci \texttt{@Table}, která určuje název tabulky v~databázi, a~anotaci \texttt{@Id}, která určuje primární klíč tabulky.
Pro kontrolu je zde anotace \texttt{ToString}, která slouží pro převod objektu na řetězec a anotace \texttt{@Data} 
z~knihovny \texttt{@Lombok}, která generuje gettery a settery pro jednotlivé atributy.
Pro každé atributy, které jsou ve výsledné databázi jako JSON je pro sloupec použita anotace \texttt{@JdbcTypeCode(SQLType.JSON)}.


\subsection{Repositáře}
Repositáře jsou rozhraní, která dědí z~\texttt{JpaRepository} a~poskytují metody pro práci s~databází.
Mezi tyto metody patří například \texttt{findAll}, \texttt{findById}, \texttt{save} a~\texttt{delete}.
Repositáře jsou označeny anotací \texttt{@Repository}, která určuje, že se jedná o~komponentu pro práci s~databází.

\subsection{Service třídy}
\label{sec:serviceTridy}
A nakonec jsou zde Service třídy, které obsahují logiku pro práci s~databází a~komunikaci s~repositáři.
Tyto třídy jsou označeny anotací \texttt{@Service}, která určuje, že se jedná o~komponentu pro práci s~databází.
Zde se provádí veškerá logika před uložení do databáze:

\newpage

\begin{itemize}
    \item Kontrola, zda Dto má primární klíč.
    \item Kontrola, zda jsou cizí klíče validní a existují v~databázi.
    \item Doplnění dat do objektu, podle již existujících dat v~databázi.
    \item Úprava dat podle konfigurace.
\end{itemize}
Po provedení těchto kontrol se objekt uloží do databáze pomocí repositáře. 

\subsection{Připojení k~databázi}
O připojení k~databázi se stará třída \texttt{DatabaseSource}, která zajišťuje navázání spojení s~databází.  
Z konfiguračního souboru se načtou potřebné informace pro připojení:
\begin{itemize}
    \item \texttt{type} -- typ databáze (např. \texttt{postgresql}, \texttt{mssql}, \texttt{oracle}),
    \item \texttt{url} -- adresa databáze (např. \texttt{localhost:5432} pro PostgreSQL),
    \item \texttt{dbname} -- název databáze (např. \texttt{ruian}),
    \item \texttt{username} -- uživatelské jméno pro připojení k~databázi,
    \item \texttt{password} -- heslo pro připojení k~databázi.
\end{itemize}

Na základě těchto informací se vytvoří připojení k~databázi, tzv. \textbf{DataSource}.
Tento zdroj bude dále upravován v~jiném modulu.  
Základem \texttt{DataSource} je nastavení základních parametrů připojení.  
Vytváří se výsledný connection string, který se upravuje podle typu databáze.  
K URL se připojí název databáze, uživatelské jméno, heslo a~v~případě MSSQL také certifikát pro zabezpečené připojení.

V dalším modulu se pro \texttt{DataSource} nastavují další parametry potřebné pro přístup k~databázi a~přenos dat.  
Dále se nastaví dialekt pro správnou syntaxi SQL příkazů podle použité databáze.

\subsection{Načtení konfigurace}
\label{sec:konfigurace}
\texttt{DatabaseSource} se stará pouze o~připojení k~databázi,  
třída \texttt{AppConfig} načítá zbytek potřebné konfigurace.
V sekci \ref{sec:konfiguracni_soubor} jak bude konfigurace rozdělena do jednotlivých částí.
Zde se budou popisovat jednotlivé atributy, jejich význam, a~tedy i~finální struktura konfiguračního souboru.
\begin{enumerate}
    \item \textbf{Konfigurace úkolů pro \texttt{Quartz Scheduler}}  
    Načítá se čas ve formátu cron pro spouštění přírůstkových dat  
    a~nastavení, zda přeskočit inicializaci hlavních územních prvků a~krajů.
    \item \textbf{Seznam krajů s~příslušnými kódy}  
    Každý řádek obsahuje kraj a~jeho kód.  
    Pokud je v~konfiguraci nastaveno přeskočení inicializace krajů nebo je seznam prázdný, tento krok se přeskočí.
    \item \textbf{Dodatečná nastavení}  
    Například volba pro ignorování geometrických dat  
    (některé databáze nepodporují geometrické typy)  
    a~nastavení velikosti jednotlivých commitů, které slouží pro optimalizaci výkonu.
    \item \textbf{Nastavení zpracování jednotlivých tabulek}
    Základním parametrem je způsob zpracování tabulek: \texttt{all} nebo \texttt{selected}.  
    \begin{itemize}
        \item \textbf{all} -- všechny tabulky budou zpracovány bez ohledu na konfiguraci,
        \item \textbf{selected} -- budou zpracovány pouze tabulky výslovně uvedené v~konfiguraci.
    \end{itemize}
    \item \textbf{Konfigurace jednotlivých tabulek}  
    Pokud je nastaven režim \texttt{selected}, zpracovávají se pouze specifikované tabulky.  
    Každá tabulka může mít vlastní nastavení:
    \begin{code}{JSON}{Konfigurace jednotlivých tabulek}{label=lst:konfTabulek}
        "<table_name>": {
            "howToProcess": "all | exclude | include",
            "columns": ["<column_name>", ..., "<column_name>"]
        }
    \end{code}
    
    Možnosti zpracování:
    \begin{itemize}
        \item \textbf{all} -- všechny sloupce budou zpracovány,
        \item \textbf{exclude} -- vybrané sloupce budou ignorovány,
        \item \textbf{include} -- vybrané sloupce budou zpracovány.
    \end{itemize}

    \item \textbf{Sloupce} u jednotlivé tabulky jsou definovány jako pole řetězců s~názvy sloupců 
    ve formátu lowercase bez mezer a~speciálních znaků. Sloupce jsou odděleny čárkami.
    Sloupce nemusí být uvedeny, pokud je nastaveno \texttt{all} u~konkrétní tabulky.
\end{enumerate}

\textbf{Možné chyby při načítání konfigurace:}
\begin{itemize}
    \item Pokud je nastaveno \texttt{exclude} nebo \texttt{include}, ale nejsou uvedeny žádné sloupce \textrightarrow{} neplatná konfigurace,
    \item Sloupce nebo tabulky, které neexistují v~databázi \textrightarrow{} budou přeskočeny,
    \item Seznam krajů je povinný atribut, i~pokud je prázdný. V~takovém případě budou zpracovány pouze základní územní prvky.
\end{itemize}
