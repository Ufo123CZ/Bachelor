\section{Quartz Scheduler}
Pro synchronizaci dat a~spouštění úloh v~určitých intervalech je použit \texttt{Quartz Scheduler}.  
Ten je přímo integrován do frameworku Spring Boot jako součást jedné z~jeho knihoven.  
Samotný scheduler se spouští při startu aplikace.

\subsection{Jobs}
\textit{Joby} jsou úlohy, které se spouštějí v~určitých intervalech nebo na základě určité události.  
V~našem případě se jedná o~3 úlohy:
\begin{itemize}
    \item \texttt{InitStatAzZsj}
    \item \texttt{InitRegion}
    \item \texttt{Additions}
\end{itemize}

\subsubsection*{InitStatAzZsj}
Tato úloha se spouští při startu aplikace a~je zodpovědná za inicializaci základních územních prvků a~krajů.  
Pokud je v~konfiguraci nastaveno přeskočení inicializace těchto prvků, úloha se neprovede.  
Úloha nejprve stáhne data o~základních územních prvcích a~krajích z~API RÚIAN za poslední měsíc.  
Následně se soubor zpracuje a~data se uloží do databáze.

\subsubsection*{InitRegion}
Tato úloha se spouští jako druhá, bezprostředně po dokončení úlohy \texttt{InitStatAzZsj}.  
Pokud je v~konfiguraci nastaveno přeskočení inicializace regionů, tato úloha se přeskočí.  
Úloha ze zadané konfigurace načte všechny kraje, které byly uvedeny.  
Každý kraj má přiřazen seznam obcí, které se postupně stahují z~API RÚIAN, zpracují a~uloží do databáze.

\subsubsection*{Additions}
Tato úloha se spouští na základě nastavení v~konfiguračním souboru nebo po dokončení úlohy \texttt{InitRegion}.  
Každý den je na API RÚIAN vytvořen nový soubor s~přírůstkovými daty za poslední den.  
Úloha tento soubor stáhne, zpracuje a~uloží do databáze.  
Poté čeká na další spuštění podle časového nastavení v~konfiguraci.

\subsection{Triggers}
\textit{Triggers} jsou spouštěče, které určují, kdy se má daný \texttt{job} spustit.  
Jak bylo zmíněno výše, úlohy \texttt{InitStatAzZsj} a~\texttt{InitRegion} se spouštějí jednorázově při startu aplikace a~navazují na sebe.  
Úloha \texttt{Additions} se naproti tomu spouští pravidelně podle nastavení v~konfiguračním souboru.  
Konkrétně je čas spuštění definován v~cron formátu.

\subsubsection*{Cron formát}
Cron formát je způsob, jak určit čas spouštění úloh.  
Původně byl použit v~systémech Unix a~stal se široce rozšířeným standardem.  
Skládá se z~následujících částí:
\begin{itemize}
    \item sekundy (0--59),
    \item minuta (0--59),
    \item hodina (0--23),
    \item den v~měsíci (1--31),
    \item měsíc (1--12 nebo zkratky názvů měsíců),
    \item den v~týdnu (0--6 nebo zkratky názvů dní).
\end{itemize}
Zkratky měsíců a~dnů se uvádějí v~angličtině, typicky ve formátu tří písmen (např. \texttt{JAN}, \texttt{MON}).
\textbf{Příklady:}
\begin{itemize}
    \item \texttt{0 0 2 * * ?} -- úloha se spustí každý den ve 2:00.
    \item \texttt{0 0 0 * * MON} -- úloha se spustí každé pondělí o~půlnoci.
    \item \texttt{30 14 3 JAN * ?} -- úloha se spustí 3.~ledna ve 14:30.
\end{itemize}