Každá databáze měla své problémy, ale všechny se nakonec podařilo vyřešit.  
V~následujících částech bude popsáno, jak byly jednotlivé databáze implementovány a~jaké problémy se objevily.

\section{PostgreSQL}
PostgreSQL se ukázala jako nejjednodušší databáze pro implementaci.  
Databáze běží na lokálním serveru v~Dockeru.  
Pro vytvoření byl stažen oficiální image z~Docker Hubu  
společně s~knihovnou PostGIS, která je potřebná pro práci s~geodaty.

\begin{lstlisting}[language=bash]
docker pull postgres:latest
\end{lstlisting}

Po stažení image byl následně vytvořen a~spuštěn kontejner pomocí docker-compose:

\begin{lstlisting}[language=bash]
docker-compose up -d
\end{lstlisting}

Databáze je inicializována skriptem `init.sql`, který vytvoří potřebné tabulky a~indexy.  
Tento skript byl napsán specificky pro databázi PostgreSQL a~pomocí něj je vytvořena databáze  
se všemi potřebnými tabulkami.

\section{Microsoft SQL Server}
Microsoft SQL Server byla druhá databáze, která byla implementována.  
Při stahování oficiálního image z~Docker Hubu se objevily problémy.  
Po úspěšném stažení image `mssql/server:2017` byl kontejner spuštěn stejně jako u~PostgreSQL  
pomocí docker-compose.

Menším rozdílem je způsob spouštění `init` skriptu, který se nespouští při inicializaci databáze  
přes `entrypoint`, ale až následně pomocí příkazu `sqlcmd`.

\newpage

\section{Oracle Database}
Oracle byla poslední databáze, která byla implementována.  
Při stahování oficiálního image z~Docker Hubu došlo k~problémům.  
Když se však image podařilo stáhnout a~kontejner spustit, databáze nefungovala správně.  
Nezbývalo nic jiného než stáhnout Oracle XE a~nainstalovat databázi na lokální stroj.  
Byla stažená verze 21c Express Edition, která je zdarma pro osobní použití.
Tato verze byla stažena z~oficiálních stránek Oracle 
\url{https://www.oracle.com/database/technologies/appdev/xe/quickstart.html}.
Konkrétně byla stažena verze pro Windows 64-bit.
Po nainstalování byla v databázi vytvořeno schéma \textbf{XEPDB1}, které je výchozím 
schématem pro databázi XE. V tomto schématu byl vytvořen uživatel \textit{ruian\_user} s~heslem \textit{12345},
který nadále bude přistupovat do databáze.
V GUI DBeaver by následně použit inicializační skript \textit{init.sql}, 
který vytvoří potřebné tabulky.


\section{Odlišnost ve skriptech pro tvorbu databází}
Každá databáze měla vlastní skript pro inicializaci databáze.  
Hlavní rozdíl byl v~syntaxi SQL příkazů.
Konkrétně se jednalo o~rozdíly v~datových typech, které byly pro každou databázi odlišné viz Tabulka \ref{tab:datove_typy}.

\renewcommand{\arraystretch}{1.3}
\begin{table}[!h]
\centering
\caption{Datové typy v různých databázích}
\label{tab:datove_typy}
    \begin{tabular}{|l|l|l|l|}
        \hline
         & \textbf{Oracle} & \textbf{PostgreSQL} & \textbf{MSSQL} \\ \hline
        \textbf{Integer}     & NUMBER           & INTEGER           & INT               \\ \hline
        \textbf{Long}        & NUMBER(19)       & BIGINT            & BIGINT            \\ \hline
        \textbf{DateTime}    & DATE             & TIMESTAMP         & DATETIME          \\ \hline
        \textbf{String}      & VARCHAR2(length) & VARCHAR(length)   & NVARCHAR(length)  \\ \hline
        \textbf{JSON}        & JSON             & JSONB             & NVARCHAR(MAX)     \\ \hline
        \textbf{Boolean}     & NUMBER(1)        & BIT               & BIT               \\ \hline
        \textbf{Geometry}    & SDO\_GEOMETRY    & GEOMETRY          & GEOMETRY          \\ \hline
        \textbf{Binary}      & BLOB             & BYTEA             & VARBINARY         \\ \hline
    \end{tabular}
\end{table}

Dalším rozdílem byly také cizí klíče (foreign key), které byly v~každé databázi definovány odlišně.
Zatím co v~PostgreSQL\ref{lst:sql_postgres_fk} a~MSSQL\ref{lst:sql_mssql_fk} se cizí klíče definují pomocí příkazu \texttt{REFERENCES}, a~\texttt{FOREIGN KEY} přímo u sloupce,
v~Oracle\ref{lst:sql_oracle_fk} se musí nejprve definovat sloupec a~poté se cizí klíč definuje pomocí příkazu \texttt{CONSTRAINT} na konci tabulky nebo po určitém sloupci.
\texttt{CONSTRAINT} navíc vytváří jméno vazby pro cizí klíč. Dále pokračuje podobně jako v~PostgreSQL a~MSSQL.

\newpage

\begin{code}{SQL}{PostgreSQL definice cizího klíče}{lst:sql_postgres_fk}
    <column_name> INTEGER REFERENCES <table>(<column_name>)
    <column_name> BIGINT REFERENCES <table>(<column_name>)
\end{code}

\begin{code}{SQL}{MSSQL definice cizího klíče}{lst:sql_mssql_fk}
    <column_name> INT FOREIGN KEY REFERENCES 
        <table>(<column_name>)
    <column_name> BIGINT FOREIGN KEY REFERENCES 
        <table>(<column_name>)
\end{code}

\begin{code}{SQL}{Oracle definice cizího klíče}{lst:sql_oracle_fk}
    <column_name> <column_type>,
    CONSTRAINT <constraint_name> FOREIGN KEY (<column_name>) 
        REFERENCES <table>(<column_name>)
\end{code}