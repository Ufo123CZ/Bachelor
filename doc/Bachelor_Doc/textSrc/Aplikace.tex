\chapter{Aplikace}
Aplikace je prostředníkem mezi RÚIAN VFR a cílovou databází.
Hlavním úkolem je stažení dat z~RÚIAN VFR, jejich přečtení, zpracování a následné uložení do cílové databáze.
Aplikace bude psána v~jazyce Java. Bude třeba zajistit funkce pro zpracování a uložení dat.

\section{Stahování dat}
Stahování dat bude zajištěno pomocí knihovny \textit{Apache HttpClient}, která je 
součástí balíčku \textit{Apache HttpComponents}. 
Tato knihovna umožňuje snadné a efektivní stahování dat z~webových stránek a API. 
V~rámci této aplikace bude použita k~stahování dat z~RÚIAN VFR.
Data budou stažena jako ZIP soubor, který bude následně rozbalen a zpracován.
Soubory budou uloženy do dočasného adresáře, který bude po zpracování smazán z~důvodu úspory místa na disku.

\section{Zpracování dat}
Zpracování dat se bude skládat z~několika částí:
přečtení a mapování dat do objektů, které budou následně uloženy do databáze.
Pro čtení dat bude třeba parser XML souborů, který přečte data a převede je do objektů.
Vzhledem k~velikosti dat bude třeba zajistit efektivní zpracování, aby nedocházelo 
k~přetížení paměti a CPU.
Pro čtení je možné využít knihovnu \textit{Jackson} nebo \textit{StAX}.

\begin{enumerate}
    \item \textbf{Jackson} -- knihovna pro zpracování JSON a XML dat. Umožňuje snadné mapování 
    objektů na JSON a XML a naopak. Je velmi rychlá a efektivní, ale může být složitější na použití.
    \item \textbf{StAX} -- knihovna pro zpracování XML dat. Umožňuje 
    čtení a zápis XML dat pomocí událostí. Je velmi rychlá a efektivní, ale může být 
    složitější na použití.
\end{enumerate}

\newpage

Následně bude třeba vytvořit objekty pro mapování dat do objektů.
Tyto objekty budou mít stejnou strukturu jako data z~RÚIAN VFR.
Pro mapování bude využita knihovna \textit{JPA} (Java Persistence API), která umožňuje snadné 
mapování objektů na databázové tabulky a naopak.
Pro tuto technologii je třeba vytvořit databázové entity, které budou mít stejnou strukturu jako
v~databázi. Dále bude třeba vytvořit repozitáře pro práci s~databází (CRUD operace).
A nakonec bude třeba vytvořit služby, které budou sloužit pro práci s~repozitáři a pro zpracování dat.

Před uložením do databáze bude třeba provést validaci dat, aby nedocházelo k~chybám při
ukládání. Je třeba zajistit, aby se zabránilo ukládání dat, která jsou neplatná nebo nekompletní.
Možné chyby při validaci dat:
\begin{itemize}
    \item Chybějící primární klíče
    \item Nevalidní cizí klíče
\end{itemize}

\section{Komunikace s~databází}
Pro ukládání dat je třeba se nejprve připojit k~databázi.
Pro připojení k~databázi bude použita knihovna \textit{JDBC} (Java Database Connectivity), která
umožňuje připojení k~různým databázím pomocí standardního API.
Pro jednotlivé databáze budou použity různé JDBC ovladače, které umožňují připojení k~vybraným databázím.

Pro připojení k~databázi bude třeba vytvořit konfigurační soubor, který bude obsahovat
informace o~připojení k~databázi.