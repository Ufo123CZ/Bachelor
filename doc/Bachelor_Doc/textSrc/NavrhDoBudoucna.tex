\label{cha:NavrhDoBudoucna}
Na této aplikaci je stále co přidávat, vylepšovat a~upravit.
Proto se v následujících sekcích pokusím shrnout, co by se 
dalo do budoucna vylepšit a přidat.

\section*{GUI pro aplikaci}
Jediný uživatelský pohled na aplikaci je v~současnosti zajištěn pomocí 
logu, který je generován v~textovém formátu. Jedná se pouze o výpis
informací o~prováděných úkonech, které aplikace dělá.
Bylo by dobré přidat uživatelské rozhraní, které by bylo schopné
zobrazit informace o~prováděných úkonech v~reálném čase.


\section*{GUI pro nastavení konfigurace}
Při psaní konfigurace uživatelem je možné že dojde k~chybě.
V~současnosti je možné že uživatel udělá chybu a~aplikace se 
se nespustí. Bylo bz užitečné vytvořit nějakou další aplikaci nebo nějaké GUI,
při spuštění aplikace, ve kterém bude možné nastavit konfiguraci aplikace.
Tímto způsobem by bylo možné uživateli pomoci s~nastavením aplikace
a~zabránit chybám, které by mohly nastat při špatném nastavení aplikace.


\section*{Optimalizace}
Aplikace je v~současnosti napsána tak, že se snaží aby byla schopná
zpracovat jakákoliv data z~RÚIANu. Ovšem rychlostně to není
úplně ideální. Hlavním kamenem úrazu je především kontrola
a~porovnání nových dat s~těmi, které jsou už v~databázi.
Bylo by dobré nějak zefektivnit tento proces, aby se obecně zrychlil
celý proces zpracování dat.


\section*{Refektoring kódu}
Tenhle oddíl vychází trochu z~předchozího. Aplikace je napsána
tak aby byla schopná zpracovat jakákoliv data z~RÚIANu.
Bylo bz proto dobré udělat refaktoring kódu, aby se zjednodušil
a zefektivnil celý proces zpracování dat. Service třídy, Dto 
by se určitě upravit pomocí nějakého generického rozhraní.
Využití abstraktních tříd a~rozhraní by mělo celý kód velmi
zjednodušit.


\section*{Zpracování zbylých dat}
V současnosti aplikace zpracovává pouze základní datové sady.
To znamená zpracování zbývajících geometrických dat, které jsou
v~RÚIANu k~dispozici. Dále zakomponovat obrázky (binární data),
které se u některých objektů vyskytují.
Databáze je v~současnosti připravena na zpracování těchto dat,
ale aplikace nemá implementovanou logiku pro jejich zpracování.


\section*{Podpora pro dalších databázových systémů}
Rozšíření aplikace o~další databázové systémy by bylo mohla
býti samozřejmostí. V~současnosti je aplikace napsána tak,
že je schopná pracovat pouze s~PostgreSQL, MS SQL a~Oracle databázemi.
Bylo by dobré přidat podporu pro další databázové systémy.