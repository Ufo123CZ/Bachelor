\chapter{Konfigurace}
Konfigurace neboli nastavení je důležitou součástí každé aplikace.
V případě této aplikace se jedná o~nastavení při jejím spuštění.

\section{Vhodný formát}
Je třeba zajistit, aby aplikace byla schopna načíst konfigurační soubor a~podle něj se správně nakonfigurovat.
Je tedy nutné zvolit vhodný formát tohoto souboru.
Existuje několik možností, z~nichž je třeba vybrat tu nejvhodnější:

\begin{itemize}
    \item \textbf{XML (XSD)} -- Extensible Markup Language
    \item \textbf{JSON} -- JavaScript Object Notation
    \item \textbf{YAML} -- YAML Ain't Markup Language
    \item \textbf{INI} -- Initialization File
\end{itemize}

\subsection{XML}
\cite{cisco_xml_json_yaml}
XML je značkovací jazyk, na rozdíl od JSON a~YAML.
Jedná se o~textový formát, který však není tak snadno čitelný pro člověka jako JSON nebo YAML.
Podporuje víceúrovňovou strukturu a~komentáře.
Dále umožňuje použití schémat, což dovoluje definovat strukturu a~typy dat.
XML je však zbytečně složité a~náročné na čtení i~psaní.
Navíc je čtení velmi pomalé a~náročné na výkon.

\subsection{JSON}
\cite{cisco_xml_json_yaml}
JSON je formát pro výměnu dat, který je snadno čitelný jak pro člověka, tak pro počítač.
Jedná se o~textový formát, který se často používá k~přenášení dat mezi serverem a~klientem.
Podporuje víceúrovňovou strukturu, ale nepodporuje komentáře.
Je však velmi rozšířený a~podporovaný většinou programovacích jazyků.
Čtení i~zápis dat je jednoduchý a~rychlý.

\subsection{YAML}
\cite{cisco_xml_json_yaml}
YAML je formát pro serializaci dat, který je čitelností a~strukturovaností podobný JSON.
Oproti JSON podporuje komentáře.
Pro člověka však může být YAML složitější a~náročnější na psaní.
Rychlost čtení je srovnatelná s~JSON, ale zápis je pomalejší.

\subsection{INI}
INI je formát konfiguračních souborů, který je snadno čitelný pro člověka.
Podporuje komentáře i~víceúrovňovou strukturu.
Je však často nejednotný a~obtížně se s~ním pracuje.
Používá se spíše pro jednodušší konfigurační soubory a~není vhodný pro složitější aplikace.

\subsection{Závěr výběru}
Vzhledem k~tomu, že aplikace bude obsahovat mnoho funkcionalit a~bude vyžadovat nastavení řady parametrů, je třeba zvolit vhodný formát.
Prozatímní verze aplikace si žádá formát, který bude čitelný, snadno rozšiřitelný a~dostatečně flexibilní.
Proto je nejvhodnější volbou formát \textbf{JSON}.
JSON je snadno čitelný a~zápis je přehledný, podporuje víceúrovňovou strukturu a~je široce rozšířený.
Přestože nepodporuje komentáře, lze jeho strukturu snadno pochopit a~rozšířit o~další parametry.
Zároveň je JSON podporován většinou programovacích jazyků, což zajišťuje snadnou integraci.

\newpage

\section{Obsah konfiguračního souboru}
Konfigurační soubor bude obsahovat nastavení aplikace.
Jedná se o~nastavení databáze, aplikace, plánovače a~parametry pro stahování dat.
Konkrétně se jedná o:

\begin{enumerate}
    \item \textbf{Databáze} -- Nastavení připojení k~databázi:
    \begin{itemize}
        \item Typ databáze -- MSSQL, PostgreSQL, Oracle
        \item Connection string -- připojovací řetězec
        \item Uživatelské jméno -- přihlašovací jméno do databáze
        \item Heslo -- heslo do databáze
        \item Název databáze -- cílová databáze pro ukládání dat
    \end{itemize}

    \item \textbf{Nastavení plánovače} -- Nastavení plánovače, který bude stahovat data:
    \begin{itemize}
        \item Interval -- interval stahování dat
        \item Přeskočení -- možnost přeskočit naplánované stahování
    \end{itemize}

    \item \textbf{Parametry pro stahování dat} -- Parametry pro stahování dat z~webové služby:
    \begin{itemize}
        \item Seznam krajů -- výčet krajů, pro které se budou data stahovat
        \item Stahovat geometrii -- ANO/NE (např. kvůli absenci geometrických typů v~Oracle XE)
        \item Velikost commitů -- počet záznamů na jeden commit do databáze
        \item Konkrétní tabulky, sloupce a~způsob práce s~vybranými daty
    \end{itemize}
\end{enumerate}

Toto nastavení bude uloženo v~konfiguračním souboru, který bude načten při startu aplikace.
Bude tedy třeba vytvořit třídu, která zajistí načtení tohoto souboru a~zpracování jeho obsahu.