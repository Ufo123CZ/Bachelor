Před začátkem popisu implementace aplikace je vhodné upřesnit, jaký je její účel.
Cílem aplikace je stahování dat z~API RÚIAN a jejich následné zpracování.
Výsledkem bude projekce těchto dat do databáze, která bude sloužit jako zdroj pro další zpracování.
Aplikace je napsána v~programovacím jazyce Java a~využívá framework Spring Boot.

\subsubsection*{Projekce databáze}
Projekce databáze představuje způsob zobrazení dat z~jedné databáze do jiné.
Podle konfigurace se nastavuje úroveň projekce, na jejímž základě se data zobrazují.
Nastavení projekce je dále popsáno v~sekci~\ref{sec:konfigurace}.

\section{Architektura aplikace}
Architektura je rozdělena do několika modulů, z~nichž každý má svůj specifický úkol:
\begin{itemize}[itemsep=0pt]
    \item \textbf{main} -- hlavní modul aplikace, který obsahuje třídu \texttt{Main}, jež spouští celou aplikaci.
    \item \textbf{config} -- stará se o~konfiguraci aplikace a~její inicializaci.
    \item \textbf{download} -- zajišťuje stahování dat z~API RÚIAN a~následné zpracování těchto dat.
    \item \textbf{scheduler} -- spravuje spouštění úloh a~konfiguraci plánovače.
\end{itemize}
Všechny závislosti a knihovny jsou spravovány pomocí nástroje \texttt{Maven}.
Konfigurační soubor se nachází v adresáři \texttt{src/main/resources} a~je pojmenován \texttt{config.json}.