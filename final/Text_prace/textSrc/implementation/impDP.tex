\section{Získávání dat z~API}
Stahování dat z~API RÚIAN je realizováno pomocí tříd \texttt{VdpClient} a~\texttt{VdpDownload},
které jsou součástí modulu \texttt{download}.

\subsection{VdpDownload}
Třída~\texttt{VdpDownload} se stará o~nastavení HTTP klienta~s~důvěrou ke všem certifikátům.
Využívá se zde knihovna~\texttt{Apache HttpClient}, která je součástí Spring Boot.
V~rámci metody \texttt{init()} se konfiguruje \texttt{SSLContext}, časové limity a~případně HTTP proxy.
Výsledný klient je uložen do proměnné \texttt{client}, která se následně používá pro volání metod \texttt{tryGet()} a~\texttt{trySaveFilter()}.
Tyto metody slouží ke stažení dat ze serveru VDP nebo k~inicializaci filtru pomocí HTTP požadavku.
Před ukončením aplikace je klient uzavřen pomocí metody označené anotací \texttt{@PreDestroy}.

\subsection{VdpClient}
\texttt{VdpClient} je hlavní komponenta~pro komunikaci se službou Veřejného dálkového přístupu (VDP).
Využívá celkem tři URL pro přístup k~seznamům souborů a~další tři URL pro stahování jednotlivých datových listů.
Adresy jsou většinou nastaveny v~konstantách této třídy, s~výjimkou jedné, která si URL generuje dynamicky podle data~předchozího dne.
Pro samotné stahování dat používá instanci třídy \\ \texttt{VdpDownload}.
Třída~obsahuje především tři metody, které jsou volány z~dříve popsaných \texttt{jobů}:

\newpage

\begin{itemize}
    \item \texttt{zpracovatStatAzZsj()} -- zpracovává pouze jeden konkrétní soubor, se základními územními prvky,
    \item \texttt{getListLinksObce()} -- zpracovává sadu souborů, které obsahují seznamy obcí podle jednotlivých krajů,
    \item \texttt{getAdditions()} -- generuje datum pro přírůstková data a~stahuje soubor s~přírůstkovými daty.
\end{itemize}

Každá z~těchto metod nejprve stáhne textový soubor, který obsahuje seznam dostupných datových souborů.
Tento seznam se následně parsuje a~získané odkazy se využijí ke stažení souborů.
Metody \texttt{zpracovatStatAzZsj()} a~\texttt{getAdditions()} stahují pouze jeden konkrétní soubor,
zatímco \texttt{getListLinksObce()} stahuje všechny soubory dostupné v~seznamu.

Z~tohoto důvodu byla~doplněna~metoda~\texttt{downloadFilesFromLinks()}, která umožňuje stáhnout všechny soubory uvedené v~daném seznamu.
Získaná data~jsou obvykle ve formátu ZIP, jsou automaticky rozbalena~a~předána~jako \texttt{InputStream} dalším komponentám
prostřednictvím funkčního rozhraní \texttt{Consumer}.

Třída~rovněž implementuje opakování požadavků v~případě selhání, logování a~čištění dočasných souborů.

\section{Zpracování dat}
Jak bylo zmíněno výše, data~jsou zpracovávána~pomocí třídy \texttt{VdpClient}
a~pomocí funkčního rozhraní \texttt{Consumer} se předávají dalším komponentám.
A~to právě komponentě \texttt{VdpParser}, která se stará o~parsování XML souboru v předaném \texttt{InputStream}
a~následné ukládání do databáze.

Na~začátku se inicializuje \texttt{XMLStreamReader}, který se stará o~parsování XML souboru.
Reader je vytvořen pomocí třídy \texttt{XMLInputFactory}, která je součástí \\ knihovny \texttt{StAX}.

Původně byl použit \texttt{DocumentBuilder}, ale ten byl nahrazen \texttt{StAX} parserem.
Celé soubory si nejprve načítal do paměti a~poté je parsoval, po prvcích (\texttt{Node}).
To bylo v~případě malých souborů (příklad při zpracování malých obcí).
Problém nastal při zpracování velkých souborů (např. obec Praha),
Soubor dosahoval velikosti přes 1 GB a~bylo potřeba~jej zpracovat po částech.
Je ale velmi náročné dělit XML bez rozbití struktury souboru.
Proto se později přešlo na~\texttt{StAX} parser, který zpracovává XML po prvcích a~nepotřebuje celou strukturu XML.
\texttt{StAX} parser je mnohem efektivnější a~umožňuje zpracovávat velké soubory bez nutnosti načítat je celé do paměti.
Zároveň čte pouze události jako je začátek a~konec elementu, což je pro zpracování dat dostačující.
Stačí tedy nadefinovat jen název potřebných elementů a~tyto elementy parsovat.
U \texttt{DocumentBuilderu} bylo třeba~číst všechny Listy a~poté je zpracovávat.

\subsection{Parsing dat}
Jak už bylo řečeno výše, data~jsou parsována~ve třídě \texttt{VdpParser} s použitím \\ \texttt{XMLStreamReader}.
Zároveň ve stejném modulu se nachází i~třída~\texttt{VdpParserConsts}, která obsahuje konstanty pro názvy jednotlivých elementů.
Při čtení XML se nejprve čte hlavička~souboru. Ta~obsahuje nepotřebné informace, které se ignorují.
Dále se čtou jednotlivá data~počínající elementem \texttt{vf:Data}.
Nejprve je potřeba~ale rozeznat o jaký element se jedná.
\texttt{XMLStreamReader} rozeznává několik eventů a~u~každého eventu se provádí jiná akce a~získává jiná informace.
Eventy, které aplikace rozeznává jsou uvedeny v~tabulce \ref{tab:xmlStreamReader}.
Příkladem, pokud se vyskytne event \\ \texttt{START\_ELEMENT}, tak z~něj můžu získat název elementu a~jeho atributy.
Pokud se vyskytne event \texttt{CHARACTERS}, tak z~něj získám text uvnitř elementu, ale pokud se pokusím získat jméno elementu vyhodí to výjimku.

Cílem je tedy číst data~v cyklu, dokud nenarazím na~konec elementu \texttt{vf:Data}.
V průběhu čtení se nachází další důležité elementy, které určují, co se zrovna~čte za~objekt.
Jeden soubor může rozeznávat až 19 různých objektů, které se parsují do různých DTO objektů.
Každý jeden objekt podléhá jinému seznamu objektů.
Je tedy potřeba rozeznat, kdy začíná list a~kdy objekt \ref{tab:seznamObjektu}.

\begin{table}[!h]
    \centering
    \caption{Eventy XMLStreamReader}
    \label{tab:xmlStreamReader}
    \begin{tabular}{|l|c|c|c|}
    \hline
    \multicolumn{1}{|c|}{\textbf{Event}} & \textbf{Hodnota~Event} & \textbf{Důležitá informace} & \textbf{Příklad}                                 \\ \hline
    \textit{START\_ELEMENT}              & 1                      & Název Elementu              & \textless{}vf:Data\textgreater{}                 \\ \hline
    \textit{END\_ELEMENT}                & 2                      & Název Elementu              & \textless{}/vf:Data\textgreater{}                \\ \hline
    \textit{CHARACTERS}                  & 4                      & Hodnoty                     & Data~                                            \\ \hline
    \end{tabular}
\end{table}

\begin{table}[!h]
    \centering
    \caption{Seznam objektů a~jejich názvy}
    \label{tab:seznamObjektu}
    \begin{tabular}{|c|c|c|}
    \hline
    \textbf{Název}                       & \textbf{List Počátek/konec}                    & \textbf{Objekt Počátek/Konec}                 \\ \hline
    \textit{Stát}                        & \textless{}vf:Staty\textgreater{}              & \textless{}vf:Stat\textgreater{}              \\ \hline
    \textit{Region soudržnosti}          & \textless{}vf:RegionySourdznosti\textgreater{} & \textless{}vf:RegionSourdznosti\textgreater{} \\ \hline
    \textit{VÚSC}                        & \textless{}vf:Vusc\textgreater{}               & \textless{}vf:Vusc\textgreater{}              \\ \hline
    \textit{Okres}                       & \textless{}vf:Okresy\textgreater{}             & \textless{}vf:Okres\textgreater{}             \\ \hline
    \textit{ORP}                         & \textless{}vf:Orp\textgreater{}                & \textless{}vf:Orp\textgreater{}               \\ \hline
    \textit{POU}                         & \textless{}vf:Pou\textgreater{}                & \textless{}vf:Pou\textgreater{}               \\ \hline
    \textit{Obce}                        & \textless{}vf:Obce\textgreater{}               & \textless{}vf:Obec\textgreater{}              \\ \hline
    \textit{Spravní obvod}               & \textless{}vf:SpravniObvody\textgreater{}      & \textless{}vf:SpravniObvod\textgreater{}      \\ \hline
    \textit{MOP}                         & \textless{}vf:Mop\textgreater{}                & \textless{}vf:Mop\textgreater{}               \\ \hline
    \textit{MOMC}                        & \textless{}vf:Momc\textgreater{}               & \textless{}vf:Momc\textgreater{}              \\ \hline
    \textit{Část obce}                   & \textless{}vf:CastiObci\textgreater{}          & \textless{}vf:CastiObce\textgreater{}         \\ \hline
    \textit{Katastrální území}           & \textless{}vf:KatastralniUzemi\textgreater{}   & \textless{}vf:KatastralniUzemi\textgreater{}  \\ \hline
    \textit{Parcela}                     & \textless{}vf:Parcely\textgreater{}            & \textless{}vf:Parcely\textgreater{}           \\ \hline
    \textit{Ulice}                       & \textless{}vf:Ulice\textgreater{}              & \textless{}vf:Ulice\textgreater{}             \\ \hline
    \textit{Stavební objekt}             & \textless{}vf:StavebniObjekty\textgreater{}    & \textless{}vf:StavebniObjekt\textgreater{}    \\ \hline
    \textit{Adresní místo}               & \textless{}vf:AdresniMista\textgreater{}       & \textless{}vf:AdresniMisto\textgreater{}      \\ \hline
    \textit{ZSJ}                         & \textless{}vf:Zsj\textgreater{}                & \textless{}vf:Zsj\textgreater{}               \\ \hline
    \textit{VO}                          & \textless{}vf:VO\textgreater{}                 & \textless{}vf:VO\textgreater{}                \\ \hline
    \textit{Zaniklý prvek}               & \textless{}vf:ZaniklePrvky\textgreater{}       & \textless{}vf:ZaniklyPrvek\textgreater{}      \\ \hline
    \end{tabular}
\end{table}

\newpage

\begin{code}{XML}{Příklad struktury XML}{lst:vfStruktura}
<vf:Data>
    <vf:Staty>
        <sti:Stat>
            <sti:Kod>...</sti:Kod>
        </sti:Stat>
    </vf:Staty>
</vf:Data>
\end{code}

Každý element je rozdělen na~dvě části.
Klasifikátor a~název elementu.
Klasifikátor je určen pro rozlišení jednotlivých elementů a do jaké úrovně patří.
Podle názvu elementu se určuje jaký objekt se parsuje.
V příkladu \ref{lst:vfStruktura} je uveden \texttt{vf} klasifikátor (\textit{výměnný formát}), který je nejvyšší úrovní.
Každý seznam objektů využívá určitý klasifikátor, přičemž každý objekt má přiřazený vlastní, specifický klasifikátor.
Každý objekt dále obsahuje atributy, které odpovídají tomuto klasifikátoru.
Příkladem může být dvojice \texttt{vf:Stat} a~\texttt{sti:Kod},
kde \texttt{vf} je klasifikátor pro seznam států a~\texttt{sti} je klasifikátor pro stát.

V~případě, že je atribut cizím klíčem, používá se klasifikátor objektu, na který odkazuje.
Může se také vyskytnout atribut, který představuje další tabulku nebo seznam tabulek.
V~takovém případě se pro tyto tabulky používá klasifikátor \texttt{com}.
Příklady klasifikátorů jsou uvedeny v~příkladu \ref{lst:klasifikator}.

\begin{code}{XML}{Příklad Klasifikátorů}{lst:klasifikator}
<vf:Data>
    <vf:Okresy>
        <vf:Okres>
            <oki:Kod>100</oki:Kod>
            <oki:Nazev>...</oki:Nazev>
        </vf:Okres>
    </vf:Okresy>
    <vf:Obce>
        <vf:Obec>
            <obi:Kod>...</obi:Kod>
            <obi:Nazev>...</obi:Nazev>
            <obi:Okres>
                <oki:Kod>100<oki:Kod>
            </obi:Okres>
            <obi:MluvnickeCharakteristiky>
                <com:Pad2>...</com:Pad2>
                <com:Pad3>...</com:Pad3>
            </obi:MluvnickeCharakteristiky>
        </vf:Obec>
    </vf:Obce>
</vf:Data>
\end{code}

\subsection{Atributy Objektů}
V tabulce \ref{tab:datove_typy} byly zmíněné všechny datové typy a~jejich rozdíly mezi databázemi.
Podle dokumentace RÚIAN VFR \cite{ruian_vfr}, každý atribut objektu má také svůj datový typ.
Konkrétně se jedná o~tyto datové typy:
\begin{itemize}
    \item \textbf{String} -- textový řetězec, který může obsahovat písmena, číslice a~speciální znaky.
    \item \textbf{Integer} -- celé číslo bez desetinné části.
    \item \textbf{Long} -- reálné číslo s~desetinnou částí.
    \item \textbf{Boolean} -- logická hodnota, která může nabývat hodnoty true nebo false.
    \item \textbf{Binární data} -- binární data, která budou primárně obsahovat obrázky.
    \item \textbf{DateTime} -- datum a~čas ve formátu \texttt{YYYY-MM-DDTHH:MM:SS} (příklad: 2007-12-03T10:15:30).
    \item \textbf{Kolekce} -- seznam objektů nebo hodnot, které jsou uloženy v~jednom atributu.
\end{itemize}

Datové typy jako \texttt{String}, \texttt{Integer}, \texttt{Boolean} a~\texttt{Long} jsou standardní datové typy.
\texttt{DateTime} je datový typ, který se používá pro ukládání data~a~času.
Co se týče binárních dat, tak ty budou uloženy jako \texttt{byte[]},
ale aplikace tato data~zatím nezpracovává.
Ovšem zmíněné \texttt{Kolekce} jsou označeny všechny atributy, které obsahují více hodnot nebo cizí klíče.
Kolekce budou do databáze ukládány jako \textbf{JSON}.

\subsubsection*{JSON Objekty}
Proč byly zvoleny JSON objekty? Tabulky dle specifikace nebyly vhodné pro rozdělení na~další tabulky a~vytváření cizích klíčů.
Některé kolekce totiž mohou být 1:1 nebo N:M. Proto byl zvolen JSON formát, který je vhodný pro ukládání takových to dat.
Některé kolekce jsou uloženy jako JSON Objekt a~některé jako JSON pole (Pole JSON Objektů).
Pro každou JSON kolekci byla vytvořena specifická JSON mapovací metoda. Každá tato metoda~funguje podobně jako XML parser.
Hledá elementy, které jsou uloženy v~kolekci a~převede je na~JSON objekt nebo pole.
Všechny atributy jednotlivých JSON objektů jsou uloženy v konstantách v~modulu \texttt{jsonObjects}.

\subsubsection*{Kolekce JSON Objekt}
Tyto kolekce vždy obsahují pouze jeden JSON Objekt.
Každá kolekce má svou vlastní metodu pro parsování.
Jedná se o~kolekce:
\begin{itemize}
    \item \textbf{Mluvnické Charakteristiky} -- metoda~\texttt{readMCh()},
    \item \textbf{Čísla~domovní} -- metoda~\texttt{readCislaDomovni()},
    \item \textbf{Nespravné Údaje} -- metoda~\texttt{readNespravneUdaje()},
\end{itemize}

\subsubsection*{Kolekce JSON Pole}
Tyto kolekce mohou obsahovat jednu nebo více Objektů.
Pro tyto kolekce se používá sada dvou metod. Jedna pro kolekci a~druhá pro jednotlivý objekt. 
Jedná se o kolekce:
\begin{itemize}
    \item \textbf{Bonitované díly / Bonitovaný díl} -- metoda~\texttt{readBonitovaneDily()} \\ a~\texttt{readBonitovanyDil()},
    \item \textbf{Způsoby ochrany / Způsob ochrany} -- metoda~\texttt{readZpusobyOchrany()} \\ a~\texttt{readZpusobOchrany()},
    \item \textbf{DetailniTEA} -- metoda~\texttt{readDetailniTeas()} a~\texttt{readDetailniTea()},
\end{itemize}

V přírůstkových datech se nachází i další kolekce, a~to \textbf{Nezjištěné údaje}.
Ta~je kompletně ignorována~a~neukládá se do databáze.

\subsubsection*{Kolekce s cizími klíči}
Cizí klíče jsou uloženy jako kolekce, které obsahují pouze jeden cizí klíč.
Z těchto kolekcí se tedy získává pouze Integer nebo Long.
Vzhledem k tomu, že se jedná o~cizí klíče, které odkazují na~jiné objekty,
byly vytvořeny metody pro parsování těchto cizích klíčů.
Metoda~\texttt{readFK()} a~\texttt{readFKLong} dostane jako parametr název elementu a~vrací Integer nebo Long.
Rozdělení na~Integer a~Long je z~důvodu, že je zde objekt \textbf{Parcela}, která má primární klíč jako Long.

\subsubsection*{Geometrie}
Geometrie je uložena jako kolekce obsahující až 3 možné geometrické údaje.
V základní datové sadě se nachází pouze Definiční bod specifikující střed daného objektu na~mapě.
V rozšířených se pak dodatečně nachází i~Generalizované hranice a~Originální hranice.
Je zde i~případ Definiční čáry, která je uložena~jako \texttt{LineString} nebo \texttt{MultiLineString}.
Tato geometrie se vyskytuje pouze u objektů \textbf{Ulice}.
O problematiku parsování geometrie se stará třída~\texttt{GeometryParser} v~modulu \texttt{geometry}, která parsuje jednotlivé geometrické objekty.
V současné práci se řeší pouze Definiční bod, který je uložen jako \texttt{Point} nebo \texttt{MultiPoint}.
Generalizované hranice a~Originální hranice jsou uloženy jako \texttt{Polygon} nebo \texttt{MultiPolygon}.
Stejně jako u~JSON objektů i~geometrie se čte podle eventu a~atributů v elementu.
Názvy geometrických objektů jsou uloženy jako konstanty v modulu \texttt{geometryParserConsts} s příslušným jménem k objektu.
Výstupem všech metod pro parsování geometrie je \texttt{Geometry} objekt z~knihovny \texttt{JTS}.

\subsection{Ukládání dat do databáze}
Po úspěšném parsování jedné sady objektů (např. \textbf{Obce}) nám vznikne list Dto objektů
dané sady, které se následně ukládají do databáze. Tento list se následně předává do metody \texttt{prepareAndSave()},
která se nachází v~příslušné service třídě (např. \texttt{ObecService}).
Tato metoda se stará o~přípravu a uložení dat do databáze.
Metoda není použita v~případě, pokud je v~konfiguraci nastaveno \texttt{howToProcess} \\ na~\texttt{selected}
a~není v~konfiguraci uvedena~tabulka~daného objektu.

Jak metoda~\texttt{prepareAndSave()} funguje?
Pro každý objekt se provede několik kontrol a~úprav.

\begin{itemize}
    \item \textbf{Bezpečnostní kontrola} -- slouží k~zabránění chyby nebo neplatným datům v~databázi.
    \item \textbf{Úprava~dat} -- slouží k~úpravě dat podle konfigurace a~podle již existujících dat v~databázi.
\end{itemize}

\begin{enumerate}
    \item \textit{(Bezpečnostní kontrola)} Pokud objekt neobsahuje primární klíč, kontrola~se zastaví a~po 
    zpracování všech objektů bude odstraněn ze seznamu.
    
    \item \textit{(Úprava~dat)} Když se objekt, již nachází v~databázi:
    \begin{enumerate}
        \item Pokud je v~konfiguraci u~tabulky nastaveno \texttt{howToProcess} na~\texttt{all}:
        Pokud je atribut u~nového objektu \texttt{null} a~v~databázi je \texttt{notnull}, tak se použije hodnota~z~databáze.
        \item Pokud je v~konfiguraci u~tabulky nastaveno \texttt{howToProcess} na~\texttt{include}:
        Pokud je atribut u~nového objektu atribut \texttt{notnull} a~v~konfiguraci tento atribut není uveden, tak se použije hodnota~z~databáze,
        jinak se použije hodnota~z~nového objektu.
        \item Pokud je v~konfiguraci u~tabulky nastaveno \texttt{howToProcess} na~\texttt{exclude}:
        Pokud je atribut u~nového objektu atribut \texttt{notnull} a~v~konfiguraci tento atribut je uveden, tak se použije hodnota~z~databáze,
        jinak se použije hodnota~z~nového objektu.
    \end{enumerate}

    \item \textit{(Úprava~dat)} Když se objekt ještě nenachází v~databázi:
    \begin{enumerate}
        \item Pokud je v~konfiguraci u~tabulky nastaveno \texttt{howToProcess} na~\texttt{all}:
        Objekt se uloží do databáze tak, jak je bez dodatečných úprav.
        \item Pokud je v~konfiguraci u~tabulky nastaveno \texttt{howToProcess} na~\texttt{include}:
        Objektu se dá hodnota~null u~atributů, které nejsou uvedeny v~konfiguraci.
        \item Pokud je v~konfiguraci u~tabulky nastaveno \texttt{howToProcess} na~\texttt{exclude}:
        Objektu se dá hodnota~null u~atributů, které jsou uvedeny v~konfiguraci.
    \end{enumerate}

    \item \textit{(Bezpečnostní kontrola)} Ověří, zda~objekt obsahuje cizí klíč, ověří se jeho platnost a~existence v~databázi.
    Pokud cizí klíč neexistuje, kontrola~se zastaví a~objekt se přidá na list pro odstranění a~po zpracování všech objektů
    bude odstraněn ze seznamu.
\end{enumerate}

Úprava~dat má dvě verze podle toho, zdali se jedná o~nový nebo již existující objekt.
Objekt z~databáze se vybírá selekcí podle primárního klíče.
Do paměti se načte objekt z~databáze a~podle toho se upraví nový objekt.

Během těchto kontrol se také loguje, kolik prvků z~listu bylo zpracováno.
Jedná se o~milníky \textit{25 \%}, \textit{50 \%}, \textit{75 \%} a~\textit{100 \%}.

Po úspěšném zpracování listu objektů se vypíše, kolik objektů bylo odstraněno a~nebude uloženo do databáze.

Pak přijde na~řadu ukládání dat do databáze.
To probíhá tak, že si list objektů rozdělí na~menší části podle velikosti \texttt{commitSize} z~konfigurace nebo menší.
Tento list se pak následně uloží do databáze pomocí metody \texttt{saveAll()}.

\section{Po zpracování dat}
Jakmile jsou data~úspěšně přečtena~a~uložena~do databáze, je potřeba~provést další úkony.
Ve třídě \texttt{VdpClient} u~metody \texttt{unzipContent()} se po úspěšném rozbalení souboru
a~zpracování souboru v~try bloku nachází finally blok který se postará o~vymazání dočasného souboru.
Tento blok se vykoná i~v~případě, že dojde k~výjimce a~soubor se neuloží do databáze.

Jakmile se soubor úspěšně vymaže z~disku může začít nový job nebo zpracování dalšího souboru
v~případě, že se jedná o~job \texttt{InitRegionJob}. 
Dále po zpracování regionů se zpracují přírůstková data (job \texttt{AdditionJob})