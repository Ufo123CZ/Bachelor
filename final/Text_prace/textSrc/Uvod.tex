Problematika správy uzemní identifikace a prostorových dat a~jejich synchronizace mezi různými systémy
nabývá na významu s~rostoucí digitalizací státní správy a~soukromých sektorů.
Jedním z~klíčových zdrojů těchto dat v~České republice je Registr územní identifikace,
adres a~nemovitostí (RÚIAN), který poskytuje rozsáhlé a~aktuální informace o~územních
objektech, adresách a~dalších klíčových entitách. Efektivní využití dat z~RÚIAN vyžaduje
nejen jejich přístup prostřednictvím datových služeb, ale také robustní řešení pro mapování,
konfiguraci a~synchronizaci datových struktur.

Cílem této bakalářské práce je analyzovat datové schéma registru RÚIAN a~možnosti získávání
dat prostřednictvím nabízených datových služeb. Dále bude provedena analýza a~návrh konfiguračního
řešení, které umožní nastavit úroveň přenášených územních objektů a~cílové databázové struktury.
V~rámci práce bude navržena a~implementována aplikace, která umožní pravidelnou synchronizaci dat
z~veřejné databáze RÚIAN do cílových databázových struktur s~podporou databází Oracle,
Microsoft~SQL Server a~PostgreSQL. Aplikace bude schopna provádět synchronizaci kompletních datových
sad i~přírůstkových změn podle zadané konfigurace.